\documentclass[11pt]{article}

\usepackage[T1]{fontenc}
\usepackage{mathpazo}
\usepackage{graphicx}
\usepackage{xcolor} % Allow colors to be defined
\usepackage{enumerate} % Needed for markdown enumerations to work
\usepackage{geometry} % Used to adjust the document margins
\usepackage{amsmath} % Equations
\usepackage{amssymb} % Equations
\usepackage{textcomp} % defines textquotesingle

\usepackage{upquote} % Upright quotes for verbatim code
\usepackage{eurosym} % defines \euro
\usepackage[mathletters]{ucs} % Extended unicode (utf-8) support
\usepackage[utf8x]{inputenc} % Allow utf-8 characters in the tex document
\usepackage{fancyvrb} % verbatim replacement that allows latex
\usepackage{grffile} % extends the file name processing of package graphics
                         % to support a larger range

% The hyperref package gives us a pdf with properly built
% internal navigation ('pdf bookmarks' for the table of contents,
% internal cross-reference links, web links for URLs, etc.)
\usepackage{hyperref}
\usepackage{longtable} % longtable support required by pandoc >1.10
\usepackage{booktabs}  % table support for pandoc > 1.12.2
\usepackage[inline]{enumitem} % IRkernel/repr support (it uses the enumerate* environment)
\usepackage[normalem]{ulem} % ulem is needed to support strikethroughs (\sout)
                                % normalem makes italics be italics, not underlines

\geometry{verbose,tmargin=1.5cm,bmargin=1.5cm,lmargin=1.5cm,rmargin=1.5cm}

\begin{document}

\title{Perhitungan struktur elektronik berdasarkan teori fungsional
    kerapatan dengan fungsi Lagrange: Implementasi awal}
\author{Fadjar Fathurrahman}
%\address{Program Studi Teknik Fisika dan Pusat Penelitian Nanosains dan
%Nanoteknologi, Institut Teknologi Bandung}
\date{}
\maketitle

\section{Pendahuluan}

Peran komputasi dalam nanosains dan nanoteknologi.

Teori fungsional kerapatan (\emph{density functional theory}).

Peran teori fungsional kerapatan

\section{Teori}

Berdasarkan teori fungsional kerapatan,
energi total dari sistem yang terdiri dari elektron yang berinteraksi
dengan suatu potential eksternal $V_{\mathrm{ext}}(\mathbf{r})$
dapat dinyatakan sebagai
\begin{equation}
E[\rho(\mathbf{r})] = 
-\frac{1}{2}\sum_{i}
\int\mathrm{d}\mathbf{r}\,
f_{i} \, \psi_{i}(\mathbf{r}) \nabla^2 \psi_{i}(\mathbf{r}) +
\int\mathrm{d}\mathbf{r}\,
\rho(\mathbf{r}) V_{\mathrm{ext}}(\mathbf{r}) +
\frac{1}{2}\int\mathrm{d}\mathbf{r}\,\mathrm{d}\mathbf{r}'\,
\frac{\rho(\mathbf{r})\rho(\mathbf{r}')}{\left|\mathbf{r}-\mathbf{r}'\right|} +
E_{\mathrm{xc}}[\rho(\mathbf{r})]
\end{equation}

LDA XC:
\begin{equation}
E_{\mathrm{xc}}[\rho(\mathbf{r})] =
\int\mathrm{d}\mathbf{r}\,
\varepsilon_{\mathrm{xc}}(\rho(\mathbf{r}))
\rho(\mathbf{r})
\end{equation}

\begin{equation}
V_{\mathrm{xc}}(\mathbf{r}) = \varepsilon_{\mathrm{xc}}(\rho(\mathbf{r}))
+ \rho(\mathbf{r})\frac{\mathrm{d}\varepsilon_{\mathrm{xc}}(\rho)}{\mathrm{d}\rho}
\end{equation}

Persamaan sentral pada teori fungsional kerapatan adalah persamaan
Kohn-Sham yang dapat ditulis sebagai berikut.
\begin{equation}
\left[
-\frac{1}{2}\nabla^2
+ V_{\mathrm{Ha}}(\mathbf{r})
+ V_{\mathrm{xc}}(\mathbf{r})
+ V_{\mathrm{ext}}(\mathbf{r})
\right]\psi_{i}(\mathrm{r}) =
\epsilon_{i} \psi_{i}(\mathrm{r})
\label{eq:KS_eq}
\end{equation}
Suku potensial pertama pada persamaan $\eqref{eq:KS_eq}$,
$V_{\mathrm{Ha}}(\mathbf{r})$ menyatakan potensial Hartree,
\begin{equation}
V_{\mathrm{Ha}}(\mathbf{r}) =
\int
\frac{\rho(\mathbf{r}')}{\left|\mathbf{r} - \mathbf{r}'\right|}
\mathrm{d}\mathbf{r}'
\end{equation}
dengan $\rho(\mathbf{r})$ adalah kerapatan (muatan) elektron
yang dapat ditulis sebagai
\begin{equation}
\rho(\mathbf{r}) = \sum_{i} f_{i}\, \psi^{*}_{i}(\mathbf{r}) \psi_{i}(\mathbf{r})
\end{equation}
Suku potensial ketida adalah $V_{\mathrm{ext}}(\mathbf{r})$
yang menyatakan potensial eksternal yang dirasakan oleh elektron.



\section{Implementasi}

Ekspansi persamaan Kohn-Sham dengan fungsi basis Lagrange:
\begin{equation}
\psi_{i}(\mathbf{r}) = \sum_{\alpha\beta\gamma}
C^{i}_{\alpha\beta\gamma} \Phi_{\alpha\beta\gamma}(\mathbf{r})
\end{equation}
dengan fungsi basis
\begin{eqnarray}
\Phi_{\alpha\beta\gamma}(\mathbf{r}) =
\phi_{\alpha}(x)\phi_{\beta}(y)\phi_{\gamma}(z)
\end{eqnarray}


\section{Perhitungan}

Potensial eksternal berupa fungsi Gaussian:
\begin{equation}
V_{\mathrm{ext}}(r) = A\exp(-\alpha r^2)
\end{equation}


\end{document}
