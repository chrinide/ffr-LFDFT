\section{Teori}

\cite{Choi2015,Choi2016}, 

Berdasarkan teori fungsional kerapatan,
energi total dari sistem yang terdiri dari elektron yang berinteraksi
dengan suatu potential eksternal $V_{\mathrm{ext}}(\mathbf{r})$
dapat dinyatakan sebagai
\begin{widetext}
\begin{equation}
E[\rho(\mathbf{r})] = 
-\frac{1}{2}\sum_{i}f_{i}
\int\mathrm{d}\mathbf{r}\,
\, \psi^{*}_{i}(\mathbf{r}) \nabla^2 \psi_{i}(\mathbf{r})
+ \int\mathrm{d}\mathbf{r}\,
\rho(\mathbf{r}) V_{\mathrm{ext}}(\mathbf{r})
+
\frac{1}{2}\int\mathrm{d}\mathbf{r}\,\mathrm{d}\mathbf{r}'\,
\frac{\rho(\mathbf{r})\rho(\mathbf{r}')}{\left|\mathbf{r}-\mathbf{r}'\right|}
+ E_{\mathrm{xc}}[\rho(\mathbf{r})]
\end{equation}
\end{widetext}

LDA XC:
\begin{equation}
E_{\mathrm{xc}}[\rho(\mathbf{r})] =
\int\mathrm{d}\mathbf{r}\,
\varepsilon_{\mathrm{xc}}(\rho(\mathbf{r}))
\rho(\mathbf{r})
\end{equation}

\begin{equation}
V_{\mathrm{xc}}(\mathbf{r}) = \varepsilon_{\mathrm{xc}}(\rho(\mathbf{r}))
+ \rho(\mathbf{r})\frac{\mathrm{d}\varepsilon_{\mathrm{xc}}(\rho)}{\mathrm{d}\rho}
\end{equation}

Persamaan sentral pada teori fungsional kerapatan adalah persamaan
Kohn-Sham yang dapat ditulis sebagai berikut.
\begin{equation}
\left[
-\frac{1}{2}\nabla^2
+ V_{\mathrm{Ha}}(\mathbf{r})
+ V_{\mathrm{xc}}(\mathbf{r})
+ V_{\mathrm{ext}}(\mathbf{r})
\right]\psi_{i}(\mathrm{r}) =
\epsilon_{i} \psi_{i}(\mathrm{r})
\label{eq:KS_eq}
\end{equation}
Suku potensial pertama pada persamaan $\eqref{eq:KS_eq}$,
$V_{\mathrm{Ha}}(\mathbf{r})$ menyatakan potensial Hartree,
\begin{equation}
V_{\mathrm{Ha}}(\mathbf{r}) =
\int
\frac{\rho(\mathbf{r}')}{\left|\mathbf{r} - \mathbf{r}'\right|}
\mathrm{d}\mathbf{r}'
\end{equation}
dengan $\rho(\mathbf{r})$ adalah kerapatan (muatan) elektron
yang dapat ditulis sebagai
\begin{equation}
\rho(\mathbf{r}) = \sum_{i} f_{i}\, \psi^{*}_{i}(\mathbf{r}) \psi_{i}(\mathbf{r})
\end{equation}
Suku potensial ketida adalah $V_{\mathrm{ext}}(\mathbf{r})$
yang menyatakan potensial eksternal yang dirasakan oleh elektron.
