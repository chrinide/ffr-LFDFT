\documentclass{beamer}

\begin{document}


\begin{frame}
\frametitle{\texttt{ffr\_LFDFT} in a nutshell}

\begin{itemize}
\item a poor man's program to do electronic structure calculation based on
\textbf{density functional theory} (\texttt{DFT})
\item basis set: \textbf{Lagrange functions} (\texttt{LF})
\item \texttt{ffr}: my initials (to distinguish between several LFDFT's)
\end{itemize}

\end{frame}


\begin{frame}
\frametitle{What can I do with \texttt{ffr\_LFDFT}?}

Not much currently:
\begin{itemize}
\item total energy calculations
\item electronic density and orbitals
\end{itemize}

\end{frame}


\begin{frame}[fragile]
\frametitle{Getting started}

\begin{itemize}

\item Clone the repository:
\begin{verbatim}
git clone https://github.com/f-fathurrahman/ffr-LFDFT ffr-LFDFT
\end{verbatim}

\item Build
\begin{verbatim}
cd ffr-LFDFT
cd src
cp ../make.inc.gfortran make.inc
make main
make postproc
\end{verbatim}

\end{itemize}

\end{frame}


\begin{frame}
\frametitle{How to use it}

It uses very similar input file with \texttt{PWSCF} from Quantum Espresso
package.

Currently, only HGH pseudopotentials are supported.

\end{frame}


\begin{frame}
\frametitle{Why wrote another package DFT code?}

Several reasons not to do it:

\begin{itemize}
\item time-consuming
\item need devotion
\end{itemize}

Several reasons to do it:

\begin{itemize}
\item to learn the "technology", knowing the nitty-gritty
\end{itemize}

\end{frame}


\begin{frame}
\frametitle{My personal motivation}

\begin{itemize}
\item To develop a code to solve Kohn-Sham equation with fast convergence with
respect to number of basis set. The convergence should be easy to control,
with only few adjustable parameters.
\item Education
\end{itemize}


\end{frame}

\end{document}

