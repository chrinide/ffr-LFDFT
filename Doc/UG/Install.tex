\section{Installation}

There is no need for installation, actually.
What is meant by installation here is compiling the library
{\tt libmain.a} and
linking the main executable of {\tt ffr-LFDFT}, namely
{\tt ffr\_LFDFT.x}

A manually written {\tt Makefile} is provided. On the topmost part of the
{\tt Makefile} you need to specify which {\tt make.inc} file you
want to use.
This {\tt make.inc} file contains definition of compiler executable,
compiler flags, and libraries used in the linking process.
Several {\tt make.inc} files that I used can be found in
the directory {\tt platform}.
You need to decide which compiler to use if there
are more than one compiler in your system.
For a typical Linux system, {\tt make.inc.gfortran} is sufficient.
You can manually edit the compiler options in the corresponding {\tt make.inc}
files.

Currently, {\tt ffr-LFDFT} is tested using the following compilers
on Linux system:
\begin{itemize}
\item GNU Fortran compiler, executable: {\tt gfortran}
\item G95 Fortran compiler, executable: {\tt g95}
\item Intel Fortran compiler, executable: {\tt ifort}
\item PGI Fortran compiler, executable: {\tt pgf90} or {\tt pgf95}
\item Sun (now part of Oracle) Fortran compiler: {\tt sunf95}
\end{itemize}

There following external libraries are required to build \ffrLFDFT
\begin{itemize}
\item BLAS
\item LAPACK
\item FFTW3
\end{itemize}

Typing the command
\begin{minted}{text}
make
\end{minted}
will build the library {\tt libmain.a} and typing
the command
\begin{minted}{text}
make main
\end{minted}
will build the main executable {\tt ffr\_LFDFT.x}.


