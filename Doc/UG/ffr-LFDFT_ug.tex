\documentclass[a4paper,11pt,fleqn]{extarticle}
\usepackage[a4paper]{geometry}
\geometry{verbose,tmargin=2cm,bmargin=2cm,lmargin=2cm,rmargin=2cm}

%\usepackage{fontspec}
%\defaultfontfeatures{Ligatures=TeX}
%\setmainfont{Linux Libertine O}
%\setmonofont{Fira Mono}

\setlength{\parindent}{0cm}

\usepackage{hyperref}
\usepackage{url}
\usepackage{xcolor}

\usepackage{amsmath}
\usepackage{amssymb}

\usepackage{minted}
%\newminted{julia}{breaklines,fontsize=\footnotesize}
\newminted{fortran}{breaklines,fontsize=\small}

\definecolor{mintedbg}{rgb}{0.95,0.95,0.95}
\usepackage{mdframed}

\BeforeBeginEnvironment{minted}{\begin{mdframed}[backgroundcolor=mintedbg]}
\AfterEndEnvironment{minted}{\end{mdframed}}

\begin{document}

\title{User Guide for {\ttfamily ffr-LFDFT}}
\author{Fadjar Fathurrahman}
\date{}
\maketitle

\tableofcontents

\section{Introduction}

Welcome to {\tt ffr-LFDFT} documentation.

{\tt ffr-LFDFT} is a poor man's program (or collection of subroutines, as of now)
to carry out electronic structure calculations based on density functional theory
and Lagrange basis set.

How to compile

How to use

input parameters ...

subroutines ... (implementation)

Add tutorial on how to use m\_LF3d module to solve Schrodinger equation
in 1d.

In LF3d periodic, only gamma-point sampling is used.

\section{Installation}

There is no need for installation, actually.
What is meant by installation here is compiling the library
{\tt libmain.a} and
linking the main executable of {\tt ffr-LFDFT}, namely
{\tt ffr\_LFDFT.x}

A manually written {\tt Makefile} is provided. On the topmost part of the
{\tt Makefile} you need to specify which {\tt make.inc} file you
want to use.
This {\tt make.inc} file contains definition of compiler executable,
compiler flags, and libraries used in the linking process.
Several {\tt make.inc} files that I used can be found in
the directory {\tt platform}.
You need to decide which compiler to use if there
are more than one compiler in your system.
For a typical Linux system, {\tt make.inc.gfortran} is sufficient.
You can manually edit the compiler options in the corresponding {\tt make.inc}
files.

Currently, {\tt ffr-LFDFT} is tested using the following compilers
on Linux system:
\begin{itemize}
\item GNU Fortran compiler, executable: {\tt gfortran}
\item G95 Fortran compiler, executable: {\tt g95}
\item Intel Fortran compiler, executable: {\tt ifort}
\item PGI Fortran compiler, executable: {\tt pgf90} or {\tt pgf95}
\item Sun (now part of Oracle) Fortran compiler: {\tt sunf95}
\end{itemize}

There following external libraries are required to build \ffrLFDFT
\begin{itemize}
\item BLAS
\item LAPACK
\item FFTW3
\end{itemize}

Typing the command
\begin{minted}{text}
make
\end{minted}
will build the library {\tt libmain.a} and typing
the command
\begin{minted}{text}
make main
\end{minted}
will build the main executable {\tt ffr\_LFDFT.x}.



\section{Usage}

\ffrLFDFT main executable, \ffrmain supports a subset of
PWSCF input file.

The following input file is for LiH molecule:
\begin{minted}{text}
&CONTROL
  pseudo_dir = '../../HGH'
  etot_conv_thr = 1.0d-6
/
&SYSTEM
  ibrav = 8
  nat = 2
  ntyp = 2
  A = 8.4668d0
  B = 8.4668d0
  C = 8.4668d0
  nr1 = 45
  nr2 = 45
  nr3 = 45
/
&ELECTRONS
  KS_Solve = 'Emin_pcg'
  cg_beta = 'DY'
  electron_maxstep = 150
  mixing_beta = 0.1
  diagonalization = 'LOBPCG'
/
ATOMIC_SPECIES
Li   3.0  Li_sc.hgh
H    1.0  H.hgh
ATOMIC_POSITIONS angstrom
Li   0.0  0.0  0.0
H    1.0  0.0  0.0
\end{minted}



\section{Kohn-Sham equation}

Within LDA, Kohn-Sham energy functional can be written as:
\begin{equation}
E_{\mathrm{LDA}}\left[\{\psi_{i}(\mathbf{r})\}\right] = 
E_{\mathrm{kin}} + E_{\mathrm{ion}} + E_{\mathrm{Ha}} + E_{\mathrm{xc}}
\end{equation}
with the following energy terms.

(1) kinetic energy:
\begin{equation}
E_{\mathrm{kin}} = -\frac{1}{2}\sum_{i_{st}}
\int f_{i_{st}}
\psi_{i_{st}}^{*}(\mathbf{r})\,\nabla^2\,\psi_{i_{st}}(\mathbf{r})
\,\mathrm{d}\mathbf{r}
\end{equation}

(2) ion-electron interaction energy:
\begin{equation}
E_{\mathrm{ion}} = \int V_{\mathrm{ion}}(\mathbf{r})\, \rho(\mathbf{r})\,
\mathrm{d}\mathbf{r}
\end{equation}

(3) Hartree (electrostatic) energy:
\begin{equation}
E_{\mathrm{Ha}} = \int \frac{1}{2}
\dfrac{\rho(\mathbf{r})\rho(\mathbf{r}')}
{\left|\mathbf{r} - \mathbf{r}'\right|}
\mathrm{d}\mathbf{r}\mathrm{d}\mathbf{r}'
\end{equation}

(4) Exchange-correlation energy (using LDA):
\begin{equation}
E_{\mathrm{xc}} = \int \epsilon_{\mathrm{xc}}\left[\rho(\mathbf{r})\right]
\rho(\mathbf{r})\,\mathrm{d}\mathbf{r}
\end{equation}

Central to the density functional theory is the so-called Kohn-Sham
equation.
This equation can be written as:
\begin{equation}
\left[
-\frac{1}{2}\nabla^2  + V_{\mathrm{KS}}(\mathbf{r})
\right] \psi_{i_{st}}(\mathbf{r}) =
\epsilon_{i_{st}}\psi_{i_{st}}(\mathbf{r})
\end{equation}
where $\epsilon{i_{st}}$ and $\psi_{i_{st}}(\mathbf{r})$ is known as Kohn-Sham
eigenvalues and eigenvectors (orbitals).
Quantity $V_{\mathrm{KS}}$ is called the Kohn-Sham potential, which can be
written as sum of several potentials:
\begin{equation}
V_{\mathrm{KS}}(\mathbf{r}) = V_{\mathrm{ion}}(\mathbf{r}) + V_{\mathrm{Ha}}(\mathbf{r})
+ V_{\mathrm{xc}}(\mathbf{r})
\label{eq:KS-pot}
\end{equation}

$V_{\mathrm{ion}}$ denotes attractive potential between ion (or atomic nuclei)
with electrons. This potential can be written as:
\begin{equation}
V_{\mathrm{ion}}(\mathrm{r}) =
\sum_{I}^{N_{\mathrm{atoms}}}
\frac{Z_{I}}{ \left| \mathbf{r} - \mathbf{R}_{I} \right| }
\end{equation}
This potential is Coulombic and has singularities
at the ionic centers. It is generally difficult to describe this
potential fully. It is common to replace the full Coulombic potential
with softer potential which is known as pseudopotential.
There are various types or flavors of pseudopotentials.
In the current implementation, ion-electron potential, $V_{\mathrm{ion}}$
is treated by pseudopotential. HGH-type pseudopotential is employed due to the
the availability of analytic forms both in real and reciprocal space.

$V_{\mathrm{Ha}}$ is the classical Hartree potential. It is defined as
\begin{equation}
V_{\mathrm{Ha}}(\mathbf{r}) = \int
\frac{\rho(\mathbf{r}')}
{\mathbf{r} - \mathbf{r}'}\,\mathrm{d}\mathbf{r}',
\end{equation}
where $\rho(\mathbf{r})$ denotes electronic density:
\begin{equation}
\rho(\mathbf{r}) = \sum_{i_{st}}^{N_{\mathrm{occ}}}
f_{i_{st}}
\psi^{*}_{i_{st}}(\mathbf{r}) \psi_{i_{st}}(\mathbf{r})
\end{equation}
Alternatively, Hartree potential can also be obtained via solving Poisson equation:
\begin{equation}
\nabla^{2} V_{\mathrm{Ha}}(\mathbf{r}) = -4\pi \rho(\mathbf{r})
\end{equation}

The last term in Equation \eqref{eq:KS-pot} is exchange-correlation potential.


\section{Lagrange basis function}

\subsection{Periodic Lagrange function}

For a given interval $[0,L]$, with $L>0$, the grid points $x_{i}$
appropriate for periodic Lagrange function are given by:

\begin{equation}
x_{i}=\frac{L}{2}\frac{2i-1}{N}
\end{equation}
with $i=1,\ldots,N$. Number of points $N$ should be an odd number.

The periodic cardinal functions $L_{i}^{\mathrm{per}}(x)$, defined
at grid point $i$ are given by:
\begin{equation}
L_{i}^{\mathrm{per}}(x)=\frac{1}{\sqrt{NL}}\sum_{n=1}^{N}\cos\left(\frac{\pi}{L}(2n-N-1)(x-x_{i})\right).
\end{equation}
The expansion of periodic function in terms of Lagrange functions:
\begin{equation}
f(x)=\sum_{i=1}^{N}c_{i}L_{i}^{\mathrm{per}}(x)
\end{equation}
with expansion coefficients $c_{i}=\sqrt{L/N}f(x_{i})$. When doing
variational calculation, the cofficients $c_{i}$ are the variational
parameters. The actual function values $f(x_{i}$) at grid points
$x_{i}$ is obtained by $f(x_{i})=\sqrt{N/L}c_{i}$. The prefactor
is sometimes abbreviated by $h=L/N$ and is also referred to as scaling
factor.

Consider periodic potential in one dimension:
\begin{equation}
V(x+L)=V(x).
\end{equation}
Floquet-Bloch theorem states that the wave function solution for periodic
potentials can be written in the form:
\begin{equation}
\psi_{k}(x)=e^{\imath kx}\phi_{k}(x)
\end{equation}
where function $\phi_{k}(x)$ and its first derivative $\phi_{k}'(x)$
have the same periodicity as $V(x)$ and $k$ is a constant called
the crystal momentum. Substituting this expression to Schrodinger
equation we obtain:
\begin{equation}
\left[-\frac{\hbar^{2}}{2m}\left(\frac{\mathrm{d}^{2}}{\mathrm{d}x^{2}}+2\imath k\frac{\mathrm{d}}{\mathrm{d}x}-k^{2}\right)+V(x)\right]\phi_{k}(x)=E\phi_{k}(k).
\end{equation}


An alternative way of enforcing periodicity of the wave function is
to require that:
\begin{equation}
\psi_{k}(x+L)=e^{\imath kL}\psi_{k}(x).
\end{equation}
This condition follows from:
\begin{eqnarray*}
\psi_{k}(x+L) & = & e^{\imath k(x+L)}\phi_{k}(x+L)\\
 & = & e^{\imath k(x+L)}\phi_{k}(x)\\
 & = & e^{\imath kL}e^{\imath kx}\phi_{k}(x)\\
 & = & e^{\imath kL}\psi_{k}(x)
\end{eqnarray*}


Using periodic cardinal the Schrodinger equation for periodic potential
can be written as:
\begin{equation}
\sum_{j=1}^{N}\left[-\frac{\hbar^{2}}{2m}\left(D_{jl}^{(2)}+2\imath kD_{jl}^{(1)}-k^{2}\delta_{jl}\right)+V(j)\delta_{jl}\right]\phi(j)=E\phi(l)
\end{equation}
with $l=1,\ldots,N$. $D_{jl}^{(1)}$ are matrix elements of the first
derivatives:
\begin{equation}
D_{jl}^{(1)}=\begin{cases}
0 & j=l\\
-\dfrac{2\pi}{L}(-1)^{j-l}\left(2\sin\dfrac{\pi(j-l)}{N}\right)^{-1} & j\neq l
\end{cases}
\end{equation}
and $D_{jl}^{(2)}$ are matrix elements of the second derivatives,
$N'=(N-1)/2$:
\begin{equation}
D_{jl}^{(2)}=\begin{cases}
-\left(\dfrac{2\pi}{L}\right)^{2}\dfrac{N'(N'+1)}{3} & j=l\\
-\left(\dfrac{2\pi}{L}\right)^{2}(-1)^{j-l}\dfrac{\cos\left(\pi(j-l)/N\right)}{2\sin^{2}\left[\pi(j-l)/N\right]} & j\neq l
\end{cases}
\end{equation}
Note that, $D_{jl}^{(1)}$ is not symmetric, but $D_{jl}^{(1)}=-D_{lj}^{(1)}$.
Meanwhile, the second derivative matrix $D_{jl}^{(2)}$ is symetric,
i.e. $D_{jl}^{(2)}=D_{lj}^{(2)}$. With the above expressions, first
and second derivative of periodic cardinals can be expressed as
\begin{eqnarray}
\frac{\mathrm{d}}{\mathrm{d}x}L_{i}^{\mathrm{per}}(x) & = & \sum_{j=1}^{N}D_{ji}^{(1)}L_{j}^{\mathrm{per}}(x)\\
\frac{\mathrm{d}^{2}}{\mathrm{d}x^{2}}L_{i}^{\mathrm{per}}(x) & = & \sum_{j=1}^{N}D_{ji}^{(2)}L_{j}^{\mathrm{per}}(x)
\end{eqnarray}


The previous approach also can be extended to periodic potential in 3D:
\[
V(\mathbf{r})=V(x,y,z)=V\left(x+L_{x},y+L_{y},z+L_{z}\right)
\]

Using periodic LF, Schrodinger equation can be casted into the following form:
\begin{equation}
\left[-\dfrac{\hbar^{2}}{2m}\left(\nabla^{2}+2\imath\mathbf{k}\cdot\nabla-\mathbf{k}^{2}\right)+V(\mathbf{r})\right]\phi_{\mathbf{k}}(\mathbf{r})=E\ \phi_{\mathbf{k}}(\mathbf{r})
\end{equation}


\subsection{Cluster Lagrange function}

For a given interval $[A,B]$, with $B>A$, the grid points $x_{i}$
appropriate for cluster Lagrange function are given by:
\[
x_{i}=A+\frac{B-A}{N+1}i
\]
where $i=1,\ldots,N$. Number of points $N$ can be either odd or
even number.

The cluster Lagrange functions $L_{i}^{\mathrm{clu}}(x)$, defined
at grid point $i$ are given by:
\begin{equation}
L_{i}^{\mathrm{clu}}(x)=\frac{2}{\sqrt{(N+1)(B-A)}}\sum_{n=1}^{N}\sin\left(k_{n}(x_{i}-A)\right)\sin\left(k_{n}(x-A)\right).
\end{equation}
where $k_{n}=\pi n/(B-A)$. The expansion of a function $f(x)$ in
terms of cluster Lagrange functions:
\begin{equation}
f(x)=\sum_{i=1}^{N}c_{i}L_{i}^{\mathrm{clu}}(x)
\end{equation}
with expansion coefficients $c_{i}=\sqrt{(B-A)/(N+1)}f(x_{i})$. When
doing variational calculation, the cofficients $c_{i}$ are the variational
parameters. The actual function values $f(x_{i}$) at grid points
$x_{i}$ is obtained by $f(x_{i})=\sqrt{(N+1)/(B-A)}c_{i}$.

Matrix elements $D_{jl}^{(2)}$ of the second derivatives for cluster
Lagrange functions are
\begin{equation}
D_{jl}^{(2)}=\begin{cases}
-\dfrac{1}{2}\left(\dfrac{\pi}{B-A}\right)^{2}\dfrac{2(N+1)^{2}+1}{3}-\dfrac{1}{\sin^{2}\left[\pi j/(N+1)\right]} & j=l\\
-\dfrac{1}{2}\left(\dfrac{\pi}{B-A}\right)^{2}(-1)^{j-l}\left[\dfrac{1}{\sin^{2}\left[\dfrac{\pi(j-l)}{2(N+1)}\right]}-\dfrac{1}{\sin^{2}\left[\dfrac{\pi(j+l)}{2(N+1)}\right]}\right] & j\neq l
\end{cases}
\end{equation}

For free or cluster boundary condition, we don't need $D_{jl}^{(1)}$.

\section{Implementation}

In this section, we wil describe our implementation of various terms 
in Kohn-Sham equations using Lagrange basis functions.
The computer program which contains our implementation can be found in public
repository: \url{https://github.com/f-fathurrahman/ffr-LFDFT}.

\subsection{Kohn-Sham equations in Lagrange basis functions representation}

Using Lagrange basis function \ref{eq:LF_p_1d} and its extension in 3d, Kohn-Sham orbitals
at point $\mathbf{r} = (x,y,z)$ can be written as
\begin{equation}
\psi_{i_{st}}(x,y,z) = \sum_{\alpha}^{N_x} \sum_{\beta}^{N_y} \sum_{\gamma}^{N_z}
C_{\alpha\beta\gamma}^{i_{st}} L_{\alpha}(x) L_{\beta}(y) L_{\gamma}(z)
\end{equation}
%
Using this expansion, kinetic operator can be written as
\begin{align}
T_{\alpha\beta\gamma}^{\alpha'\beta'\gamma'} & = -\frac{1}{2} \sum_{i_{st}} f_{i_{st}}
\Braket{ \psi_{i_{st}} | \nabla^2 | \psi_{i} } \\
& =
-\frac{1}{2}
\sum_{i_{st}} f_{i_{st}} \sum_{\alpha\alpha'} \sum_{\beta\beta'} \sum_{\gamma\gamma'}
C^{i_{st}}_{\alpha\beta\gamma} \mathbb{L}_{\alpha\beta\gamma}^{\alpha'\beta'\gamma'}
C^{i_{st}}_{\alpha'\beta'\gamma'}
\end{align}
%
were the Laplacian matrix $\mathbb{L}_{\alpha\beta\gamma}^{\alpha'\beta'\gamma'}$
has the following form:
\begin{equation}
\mathbb{L}_{\alpha\beta\gamma}^{\alpha'\beta'\gamma'} =
D^{(2)}_{\alpha\alpha'}\delta_{\beta\beta'}\delta_{\gamma\gamma'} +
D^{(2)}_{\beta\beta'}\delta_{\alpha\alpha'}\delta_{\gamma\gamma'} +
D^{(2)}_{\gamma\gamma'}\delta_{\alpha\alpha'}\delta_{\beta\beta'}
\end{equation}
%
Specifically, for periodic Lagrange basis function $D^{(2)}_{ij}$, $i, j = \alpha, \beta, \gamma$
can be written as follows.
\begin{equation}
D^{(2)}_{ij} = -\left( \frac{2\pi}{L} \right)^2 \frac{N'}{3} \left( N' + 1 \right) \delta_{ij} \\
+ \dfrac{ \left(\dfrac{2\pi}{L}\right)^2 (-1)^{i-j}\cos\left[\dfrac{\pi(i-j)}{N}\right]}
{2\sin^2\left[\dfrac{\pi(i-j)}{N}\right]}
(1-\delta_{nn'})
\label{eq:kin1d_p}
\end{equation}
where $N' = (N-1)/2$.

The matrix representation of kinetic operator is sparse.

The remaining potential terms which are local have very simple matrix form, i.e.
diagonal. The action of potential operator to Kohn-Sham orbital at point
$(r_{\alpha\beta\gamma})$ thus can be
obtained by pointwise multiplication with the potential on that point:
\begin{equation}
V_{\mathrm{KS}}(r_{\alpha\beta\gamma}) = V_{\mathrm{ion}}(r_{\alpha\beta\gamma}) +
V_{\mathrm{Ha}}(r_{\alpha\beta\gamma}) + V_{\mathrm{xc}}(r_{\alpha\beta\gamma})
\end{equation}

\subsection{Methods to solve Kohn-Sham equations}

We implement two methods to solve the Kohn-Sham equations, namely
via the self-consistent field (SCF) iterations and
direct energy minimization.

Outline of SCF iterations:
\begin{itemize}
\item Guess density $\rho(\mathbf{r})$
\item Iterate until convergence
\begin{itemize}
\item Calculate Kohn-Sham potentials $V_{\mathrm{KS}}$ and build the Kohn-Sham
Hamiltonian $H_{\mathrm{KS}}$
\item Diagonalize $H_{\mathrm{KS}}$ to obtain $\mathrm{\psi_{i_{st}}}(\mathbf{r})$
and $\epsilon_{i_{st}}$.
\item Calculate charge density and total energy. If the calculation converges
the stop the calculation, if not iterate.
\end{itemize}
\end{itemize}

Outline of direct minimization, using 
\begin{itemize}
\item Generate guess Kohn-Sham orbitals, orthonormalize if needed.
\item Calculate charge density, build Kohn-Sham potential and calculate total energy
for this
\item Iterate until convergence:
%
\begin{itemize}
%
\item Calculate Kohn-Sham electronic gradient $\mathbf{g}_{\psi}$ and the preconditioned
gradient $\mathbf{Kg}_{\psi}$ where $\mathbf{K}$ is a preconditioner.
%
\item Calculate search direction:
\begin{equation}
\beta = \dfrac{\mathbf{g}_{\psi}^{\dagger}\mathbf{Kg}_{\psi}}
{\mathbf{g}_{\psi,\mathrm{prev}}^{\dagger}\mathbf{Kg}_{\psi,\mathrm{prev}}}
\end{equation}
If $\mathbf{g}_{\psi,\mathrm{prev}}$ is not available (first iteration) then set
$\beta = 0$.
%
\item Calculate new direction:
\begin{equation}
\mathbf{d} = 
\end{equation}
\end{itemize}
%
\end{itemize}


\end{document}


