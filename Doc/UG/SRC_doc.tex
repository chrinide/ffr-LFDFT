
\documentclass[a4paper,10pt,fleqn]{article}
\usepackage[a4paper]{geometry}
\geometry{verbose,tmargin=2cm,bmargin=2cm,lmargin=1.5cm,rmargin=1.5cm}

\setlength{\parskip}{\smallskipamount}
\setlength{\parindent}{0pt}

\usepackage{hyperref}
\usepackage{url}
\usepackage{fancyvrb}
\usepackage{xcolor}



\definecolor{mygray}{rgb}{0.95,0.95,0.95}
\usepackage{mdframed}
\BeforeBeginEnvironment{Verbatim}{\begin{mdframed}[backgroundcolor=mygray]}
\AfterEndEnvironment{Verbatim}{\end{mdframed}}


\makeatletter
\def\PY@reset{\let\PY@it=\relax \let\PY@bf=\relax%
    \let\PY@ul=\relax \let\PY@tc=\relax%
    \let\PY@bc=\relax \let\PY@ff=\relax}
\def\PY@tok#1{\csname PY@tok@#1\endcsname}
\def\PY@toks#1+{\ifx\relax#1\empty\else%
    \PY@tok{#1}\expandafter\PY@toks\fi}
\def\PY@do#1{\PY@bc{\PY@tc{\PY@ul{%
    \PY@it{\PY@bf{\PY@ff{#1}}}}}}}
\def\PY#1#2{\PY@reset\PY@toks#1+\relax+\PY@do{#2}}

\expandafter\def\csname PY@tok@gd\endcsname{\def\PY@tc##1{\textcolor[rgb]{0.63,0.00,0.00}{##1}}}
\expandafter\def\csname PY@tok@gu\endcsname{\let\PY@bf=\textbf\def\PY@tc##1{\textcolor[rgb]{0.50,0.00,0.50}{##1}}}
\expandafter\def\csname PY@tok@gt\endcsname{\def\PY@tc##1{\textcolor[rgb]{0.00,0.27,0.87}{##1}}}
\expandafter\def\csname PY@tok@gs\endcsname{\let\PY@bf=\textbf}
\expandafter\def\csname PY@tok@gr\endcsname{\def\PY@tc##1{\textcolor[rgb]{1.00,0.00,0.00}{##1}}}
\expandafter\def\csname PY@tok@cm\endcsname{\let\PY@it=\textit\def\PY@tc##1{\textcolor[rgb]{0.25,0.50,0.50}{##1}}}
\expandafter\def\csname PY@tok@vg\endcsname{\def\PY@tc##1{\textcolor[rgb]{0.10,0.09,0.49}{##1}}}
\expandafter\def\csname PY@tok@vi\endcsname{\def\PY@tc##1{\textcolor[rgb]{0.10,0.09,0.49}{##1}}}
\expandafter\def\csname PY@tok@mh\endcsname{\def\PY@tc##1{\textcolor[rgb]{0.40,0.40,0.40}{##1}}}
\expandafter\def\csname PY@tok@cs\endcsname{\let\PY@it=\textit\def\PY@tc##1{\textcolor[rgb]{0.25,0.50,0.50}{##1}}}
\expandafter\def\csname PY@tok@ge\endcsname{\let\PY@it=\textit}
\expandafter\def\csname PY@tok@vc\endcsname{\def\PY@tc##1{\textcolor[rgb]{0.10,0.09,0.49}{##1}}}
\expandafter\def\csname PY@tok@il\endcsname{\def\PY@tc##1{\textcolor[rgb]{0.40,0.40,0.40}{##1}}}
\expandafter\def\csname PY@tok@go\endcsname{\def\PY@tc##1{\textcolor[rgb]{0.53,0.53,0.53}{##1}}}
\expandafter\def\csname PY@tok@cp\endcsname{\def\PY@tc##1{\textcolor[rgb]{0.74,0.48,0.00}{##1}}}
\expandafter\def\csname PY@tok@gi\endcsname{\def\PY@tc##1{\textcolor[rgb]{0.00,0.63,0.00}{##1}}}
\expandafter\def\csname PY@tok@gh\endcsname{\let\PY@bf=\textbf\def\PY@tc##1{\textcolor[rgb]{0.00,0.00,0.50}{##1}}}
\expandafter\def\csname PY@tok@ni\endcsname{\let\PY@bf=\textbf\def\PY@tc##1{\textcolor[rgb]{0.60,0.60,0.60}{##1}}}
\expandafter\def\csname PY@tok@nl\endcsname{\def\PY@tc##1{\textcolor[rgb]{0.63,0.63,0.00}{##1}}}
\expandafter\def\csname PY@tok@nn\endcsname{\let\PY@bf=\textbf\def\PY@tc##1{\textcolor[rgb]{0.00,0.00,1.00}{##1}}}
\expandafter\def\csname PY@tok@no\endcsname{\def\PY@tc##1{\textcolor[rgb]{0.53,0.00,0.00}{##1}}}
\expandafter\def\csname PY@tok@na\endcsname{\def\PY@tc##1{\textcolor[rgb]{0.49,0.56,0.16}{##1}}}
\expandafter\def\csname PY@tok@nb\endcsname{\def\PY@tc##1{\textcolor[rgb]{0.00,0.50,0.00}{##1}}}
\expandafter\def\csname PY@tok@nc\endcsname{\let\PY@bf=\textbf\def\PY@tc##1{\textcolor[rgb]{0.00,0.00,1.00}{##1}}}
\expandafter\def\csname PY@tok@nd\endcsname{\def\PY@tc##1{\textcolor[rgb]{0.67,0.13,1.00}{##1}}}
\expandafter\def\csname PY@tok@ne\endcsname{\let\PY@bf=\textbf\def\PY@tc##1{\textcolor[rgb]{0.82,0.25,0.23}{##1}}}
\expandafter\def\csname PY@tok@nf\endcsname{\def\PY@tc##1{\textcolor[rgb]{0.00,0.00,1.00}{##1}}}
\expandafter\def\csname PY@tok@si\endcsname{\let\PY@bf=\textbf\def\PY@tc##1{\textcolor[rgb]{0.73,0.40,0.53}{##1}}}
\expandafter\def\csname PY@tok@s2\endcsname{\def\PY@tc##1{\textcolor[rgb]{0.73,0.13,0.13}{##1}}}
\expandafter\def\csname PY@tok@nt\endcsname{\let\PY@bf=\textbf\def\PY@tc##1{\textcolor[rgb]{0.00,0.50,0.00}{##1}}}
\expandafter\def\csname PY@tok@nv\endcsname{\def\PY@tc##1{\textcolor[rgb]{0.10,0.09,0.49}{##1}}}
\expandafter\def\csname PY@tok@s1\endcsname{\def\PY@tc##1{\textcolor[rgb]{0.73,0.13,0.13}{##1}}}
\expandafter\def\csname PY@tok@ch\endcsname{\let\PY@it=\textit\def\PY@tc##1{\textcolor[rgb]{0.25,0.50,0.50}{##1}}}
\expandafter\def\csname PY@tok@m\endcsname{\def\PY@tc##1{\textcolor[rgb]{0.40,0.40,0.40}{##1}}}
\expandafter\def\csname PY@tok@gp\endcsname{\let\PY@bf=\textbf\def\PY@tc##1{\textcolor[rgb]{0.00,0.00,0.50}{##1}}}
\expandafter\def\csname PY@tok@sh\endcsname{\def\PY@tc##1{\textcolor[rgb]{0.73,0.13,0.13}{##1}}}
\expandafter\def\csname PY@tok@ow\endcsname{\let\PY@bf=\textbf\def\PY@tc##1{\textcolor[rgb]{0.67,0.13,1.00}{##1}}}
\expandafter\def\csname PY@tok@sx\endcsname{\def\PY@tc##1{\textcolor[rgb]{0.00,0.50,0.00}{##1}}}
\expandafter\def\csname PY@tok@bp\endcsname{\def\PY@tc##1{\textcolor[rgb]{0.00,0.50,0.00}{##1}}}
\expandafter\def\csname PY@tok@c1\endcsname{\let\PY@it=\textit\def\PY@tc##1{\textcolor[rgb]{0.25,0.50,0.50}{##1}}}
\expandafter\def\csname PY@tok@o\endcsname{\def\PY@tc##1{\textcolor[rgb]{0.40,0.40,0.40}{##1}}}
\expandafter\def\csname PY@tok@kc\endcsname{\let\PY@bf=\textbf\def\PY@tc##1{\textcolor[rgb]{0.00,0.50,0.00}{##1}}}
\expandafter\def\csname PY@tok@c\endcsname{\let\PY@it=\textit\def\PY@tc##1{\textcolor[rgb]{0.25,0.50,0.50}{##1}}}
\expandafter\def\csname PY@tok@mf\endcsname{\def\PY@tc##1{\textcolor[rgb]{0.40,0.40,0.40}{##1}}}
\expandafter\def\csname PY@tok@err\endcsname{\def\PY@bc##1{\setlength{\fboxsep}{0pt}\fcolorbox[rgb]{1.00,0.00,0.00}{1,1,1}{\strut ##1}}}
\expandafter\def\csname PY@tok@mb\endcsname{\def\PY@tc##1{\textcolor[rgb]{0.40,0.40,0.40}{##1}}}
\expandafter\def\csname PY@tok@ss\endcsname{\def\PY@tc##1{\textcolor[rgb]{0.10,0.09,0.49}{##1}}}
\expandafter\def\csname PY@tok@sr\endcsname{\def\PY@tc##1{\textcolor[rgb]{0.73,0.40,0.53}{##1}}}
\expandafter\def\csname PY@tok@mo\endcsname{\def\PY@tc##1{\textcolor[rgb]{0.40,0.40,0.40}{##1}}}
\expandafter\def\csname PY@tok@kd\endcsname{\let\PY@bf=\textbf\def\PY@tc##1{\textcolor[rgb]{0.00,0.50,0.00}{##1}}}
\expandafter\def\csname PY@tok@mi\endcsname{\def\PY@tc##1{\textcolor[rgb]{0.40,0.40,0.40}{##1}}}
\expandafter\def\csname PY@tok@kn\endcsname{\let\PY@bf=\textbf\def\PY@tc##1{\textcolor[rgb]{0.00,0.50,0.00}{##1}}}
\expandafter\def\csname PY@tok@cpf\endcsname{\let\PY@it=\textit\def\PY@tc##1{\textcolor[rgb]{0.25,0.50,0.50}{##1}}}
\expandafter\def\csname PY@tok@kr\endcsname{\let\PY@bf=\textbf\def\PY@tc##1{\textcolor[rgb]{0.00,0.50,0.00}{##1}}}
\expandafter\def\csname PY@tok@s\endcsname{\def\PY@tc##1{\textcolor[rgb]{0.73,0.13,0.13}{##1}}}
\expandafter\def\csname PY@tok@kp\endcsname{\def\PY@tc##1{\textcolor[rgb]{0.00,0.50,0.00}{##1}}}
\expandafter\def\csname PY@tok@w\endcsname{\def\PY@tc##1{\textcolor[rgb]{0.73,0.73,0.73}{##1}}}
\expandafter\def\csname PY@tok@kt\endcsname{\def\PY@tc##1{\textcolor[rgb]{0.69,0.00,0.25}{##1}}}
\expandafter\def\csname PY@tok@sc\endcsname{\def\PY@tc##1{\textcolor[rgb]{0.73,0.13,0.13}{##1}}}
\expandafter\def\csname PY@tok@sb\endcsname{\def\PY@tc##1{\textcolor[rgb]{0.73,0.13,0.13}{##1}}}
\expandafter\def\csname PY@tok@k\endcsname{\let\PY@bf=\textbf\def\PY@tc##1{\textcolor[rgb]{0.00,0.50,0.00}{##1}}}
\expandafter\def\csname PY@tok@se\endcsname{\let\PY@bf=\textbf\def\PY@tc##1{\textcolor[rgb]{0.73,0.40,0.13}{##1}}}
\expandafter\def\csname PY@tok@sd\endcsname{\let\PY@it=\textit\def\PY@tc##1{\textcolor[rgb]{0.73,0.13,0.13}{##1}}}

\def\PYZbs{\char`\\}
\def\PYZus{\char`\_}
\def\PYZob{\char`\{}
\def\PYZcb{\char`\}}
\def\PYZca{\char`\^}
\def\PYZam{\char`\&}
\def\PYZlt{\char`\<}
\def\PYZgt{\char`\>}
\def\PYZsh{\char`\#}
\def\PYZpc{\char`\%}
\def\PYZdl{\char`\$}
\def\PYZhy{\char`\-}
\def\PYZsq{\char`\'}
\def\PYZdq{\char`\"}
\def\PYZti{\char`\~}
% for compatibility with earlier versions
\def\PYZat{@}
\def\PYZlb{[}
\def\PYZrb{]}
\makeatother


\begin{document}


\section{Main program: \texttt{ffr\_LFDFT}}

This is the main program for \texttt{ffr\_LFDFT}.

Start from here if you want to learn the call sequence of the main program.
Most of the subroutines called here are wrapper subroutines which call
various core computational subroutines.

If you want to quickly do a calculation for simple system or want to use directly
the core computational subroutines, please see examples in directory \texttt{tests}.


\begin{Verbatim}[commandchars=\\\{\}]
\PY{k}{PROGRAM }\PY{n}{ffr\PYZus{}LFDFT}
  \PY{k}{IMPLICIT }\PY{k}{NONE}
\end{Verbatim}
Variables for timing stuffs
Note that, for the moment the program only does timing for overall subroutines.
It is very desirable, however, to measure timing of various steps in the program.

\begin{Verbatim}[commandchars=\\\{\}]
  \PY{k+kt}{INTEGER} \PY{k+kd}{::} \PY{n}{tstart}\PY{p}{,} \PY{n}{counts\PYZus{}per\PYZus{}second}\PY{p}{,} \PY{n}{tstop}
\end{Verbatim}

Here begins the calls of wrapper subroutines.

\begin{Verbatim}[commandchars=\\\{\}]
\PY{c}{!!}
\end{Verbatim}
\begin{itemize}

\item
The following subroutine will print some information about the program.

\begin{Verbatim}[commandchars=\\\{\}]
  \PY{k}{CALL }\PY{n}{welcome}\PY{p}{(}\PY{p}{)}
\end{Verbatim}
\item
We start timing from here.

\begin{Verbatim}[commandchars=\\\{\}]
  \PY{k}{CALL }\PY{n+nb}{system\PYZus{}clock}\PY{p}{(} \PY{n}{tstart}\PY{p}{,} \PY{n}{counts\PYZus{}per\PYZus{}second} \PY{p}{)}
\end{Verbatim}
\item
This subroutine handles reading input file, setting-up
basis set and grid points, pseudopotential-related variables, etc.

\begin{Verbatim}[commandchars=\\\{\}]
  \PY{k}{CALL }\PY{n}{setup\PYZus{}ffr\PYZus{}LFDFT}\PY{p}{(}\PY{p}{)}
\end{Verbatim}
\item
Preparation for solving Kohn-Sham equation

\begin{Verbatim}[commandchars=\\\{\}]
  \PY{k}{CALL }\PY{n}{guess\PYZus{}KS\PYZus{}solutions}\PY{p}{(}\PY{p}{)}
\end{Verbatim}
\item
This soubroutine is the driver for primary subroutines to solve Kohn-Sham
equation.

\begin{Verbatim}[commandchars=\\\{\}]
  \PY{k}{CALL }\PY{n}{do\PYZus{}KS\PYZus{}solve}\PY{p}{(}\PY{p}{)}
\end{Verbatim}
\item
Free allocated memory

\begin{Verbatim}[commandchars=\\\{\}]
  \PY{k}{CALL }\PY{n}{cleanup\PYZus{}ffr\PYZus{}LFDFT}\PY{p}{(}\PY{p}{)}
\end{Verbatim}
\item
Stop timing and display elapsed time

\begin{Verbatim}[commandchars=\\\{\}]
  \PY{k}{CALL }\PY{n+nb}{system\PYZus{}clock}\PY{p}{(} \PY{n}{tstop} \PY{p}{)}
  \PY{k}{WRITE}\PY{p}{(}\PY{o}{*}\PY{p}{,}\PY{o}{*}\PY{p}{)}
  \PY{k}{WRITE}\PY{p}{(}\PY{o}{*}\PY{p}{,}\PY{l+s+s1}{\PYZsq{}(1x,A,ES18.10,A)\PYZsq{}}\PY{p}{)} \PY{l+s+s1}{\PYZsq{}Total elapsed time: \PYZsq{}}\PY{p}{,} \PY{p}{\PYZam{}}
           \PY{n+nb}{dble}\PY{p}{(}\PY{n}{tstop} \PY{o}{\PYZhy{}} \PY{n}{tstart}\PY{p}{)}\PY{o}{/}\PY{n}{counts\PYZus{}per\PYZus{}second}\PY{p}{,} \PY{l+s+s1}{\PYZsq{} second.\PYZsq{}}
\end{Verbatim}
\item
Display goodbye message

\begin{Verbatim}[commandchars=\\\{\}]
  \PY{k}{CALL }\PY{n}{goodbye}\PY{p}{(}\PY{p}{)}
\PY{k}{END }\PY{k}{PROGRAM}
\end{Verbatim}
\end{itemize}

\section{Subroutine \texttt{setup\_ffr\_LFDFT()}}

This subroutine prepares various tasks before actually solving the Kohn-Sham
equation.



\begin{Verbatim}[commandchars=\\\{\}]
\PY{k}{SUBROUTINE }\PY{n}{setup\PYZus{}ffr\PYZus{}LFDFT}\PY{p}{(}\PY{p}{)}
  \PY{k}{USE }\PY{n}{m\PYZus{}options}\PY{p}{,} \PY{k}{ONLY} \PY{p}{:} \PY{n}{FREE\PYZus{}NABLA2}\PY{p}{,} \PY{n}{I\PYZus{}POISSON\PYZus{}SOLVE}
  \PY{k}{USE }\PY{n}{m\PYZus{}input\PYZus{}vars}\PY{p}{,} \PY{k}{ONLY} \PY{p}{:} \PY{n}{assume\PYZus{}isolated}
  \PY{k}{USE }\PY{n}{m\PYZus{}LF3d}\PY{p}{,} \PY{k}{ONLY} \PY{p}{:} \PY{n}{Npoints} \PY{o}{=}\PY{o}{\PYZgt{}} \PY{n}{LF3d\PYZus{}Npoints}
  \PY{k}{USE }\PY{n}{m\PYZus{}states}\PY{p}{,} \PY{k}{ONLY} \PY{p}{:} \PY{n}{Nstates}\PY{p}{,} \PY{p}{\PYZam{}}
                       \PY{n}{evals} \PY{o}{=}\PY{o}{\PYZgt{}} \PY{n}{KS\PYZus{}evals}\PY{p}{,} \PY{p}{\PYZam{}}
                       \PY{n}{evecs} \PY{o}{=}\PY{o}{\PYZgt{}} \PY{n}{KS\PYZus{}evecs} 
  \PY{c}{!}
  \PY{k+kt}{INTEGER} \PY{k+kd}{::} \PY{n}{Narg}   \PY{c}{! number of argument}
  \PY{k+kt}{INTEGER} \PY{k+kd}{::} \PY{n+nb}{iargc}  \PY{c}{! needed for several compilers}
  \PY{k+kt}{CHARACTER}\PY{p}{(}\PY{l+m+mi}{64}\PY{p}{)} \PY{k+kd}{::} \PY{n}{filein}
\end{Verbatim}
\begin{itemize}

\item
We first check the number of argument(s) give to the program using built-in
function \texttt{iargc()} and save the result to variable \texttt{Narg}.
Currently, we only support one argument, i.e. path to input file.
The program will stop and display error message if \texttt{Narg}
is not equal to one.

\begin{Verbatim}[commandchars=\\\{\}]
  \PY{n}{Narg} \PY{o}{=} \PY{n+nb}{iargc}\PY{p}{(}\PY{p}{)}
  \PY{k}{IF}\PY{p}{(} \PY{n}{Narg} \PY{o}{/}\PY{o}{=} \PY{l+m+mi}{1} \PY{p}{)} \PY{k}{THEN}
    \PY{k}{WRITE}\PY{p}{(}\PY{o}{*}\PY{p}{,}\PY{o}{*}\PY{p}{)}
    \PY{k}{WRITE}\PY{p}{(}\PY{o}{*}\PY{p}{,}\PY{o}{*}\PY{p}{)} \PY{l+s+s1}{\PYZsq{}ERROR: exactly one argument must be given: input file path\PYZsq{}}
    \PY{k}{STOP}
  \PY{k}{ENDIF}
\end{Verbatim}
\item
We get the actual argument using built-in subroutine \texttt{getarg()}.

\begin{Verbatim}[commandchars=\\\{\}]
  \PY{k}{CALL }\PY{n+nb}{getarg}\PY{p}{(} \PY{l+m+mi}{1}\PY{p}{,} \PY{n}{filein} \PY{p}{)}
\end{Verbatim}
\item
We read the input file using subroutine \texttt{read\_input()}

\begin{Verbatim}[commandchars=\\\{\}]
  \PY{k}{CALL }\PY{n}{read\PYZus{}input}\PY{p}{(} \PY{n}{filein} \PY{p}{)}
\end{Verbatim}
\item
The following subroutine will initialize global variables related to basis function
and grid points, molecular or crystalline structures, and pseudopotentials.

\begin{Verbatim}[commandchars=\\\{\}]
  \PY{k}{CALL }\PY{n}{setup\PYZus{}from\PYZus{}input}\PY{p}{(}\PY{p}{)}
\end{Verbatim}
\item
Various options, such as convergence criteria, choice of algorithms, etc which are
given in the input file, will be converted to internal variables (mostly defined in
module \texttt{m\_options}).

\begin{Verbatim}[commandchars=\\\{\}]
  \PY{k}{CALL }\PY{n}{setup\PYZus{}options}\PY{p}{(}\PY{p}{)}
\end{Verbatim}
\item
The following calls will output information about molecular or crystalline
structures, pseudopotentials, and basis function and grid points.

\begin{Verbatim}[commandchars=\\\{\}]
  \PY{k}{CALL }\PY{n}{info\PYZus{}atoms}\PY{p}{(}\PY{p}{)}
  \PY{k}{CALL }\PY{n}{info\PYZus{}PsPot}\PY{p}{(}\PY{p}{)}
  \PY{k}{CALL }\PY{n}{info\PYZus{}LF3d}\PY{p}{(}\PY{p}{)}
\end{Verbatim}
\item
This subroutine initialize nonlocal pseudopotential projectors. It
must be called after variables from \texttt{m\_LF3d} are initialized as they are
defined on grid points.

TODO/FIXME: To be consistent, this call should be made in pseudopotential setup.
(probably via subroutine \texttt{setup\_from\_input()})

\begin{Verbatim}[commandchars=\\\{\}]
  \PY{k}{CALL }\PY{n}{init\PYZus{}betaNL}\PY{p}{(}\PY{p}{)}
\end{Verbatim}
\item
This call will determined number of occupied and unoccupied states.
It will also initialize occupation numbers.

Note: Memories for eigenvectors and eigenvalues of KS equations are not
allocated here.

\begin{Verbatim}[commandchars=\\\{\}]
  \PY{k}{CALL }\PY{n}{init\PYZus{}states}\PY{p}{(}\PY{p}{)}
\end{Verbatim}
\item
Allocate KS eigenvectors and eigenvalues.

This step should be done in some wrapper subroutine, however, for current use-case
this is suffice.

\begin{Verbatim}[commandchars=\\\{\}]
  \PY{k}{ALLOCATE}\PY{p}{(} \PY{n}{evecs}\PY{p}{(}\PY{n}{Npoints}\PY{p}{,}\PY{n}{Nstates}\PY{p}{)}\PY{p}{,} \PY{n}{evals}\PY{p}{(}\PY{n}{Nstates}\PY{p}{)} \PY{p}{)}
\end{Verbatim}
\item
This call will initialize and calculate structure factor $S_{f}(\mathbf{G})$.
This is required for periodic LF.

TODO/FIXME: This call should be made in \texttt{setup\_from\_input} or anther
subroutine.

\begin{Verbatim}[commandchars=\\\{\}]
  \PY{k}{CALL }\PY{n}{init\PYZus{}strfact\PYZus{}shifted}\PY{p}{(}\PY{p}{)}
\end{Verbatim}
\item
Ewald energy is calculated here.

FIXME: Ewald energy calculation should be called everytime atomic positions
are updated, for example in geometry optimization or molecular dynamics.

\begin{Verbatim}[commandchars=\\\{\}]
  \PY{k}{IF}\PY{p}{(} \PY{n}{assume\PYZus{}isolated} \PY{o}{==} \PY{l+s+s1}{\PYZsq{}sinc\PYZsq{}} \PY{p}{)} \PY{k}{THEN}
    \PY{k}{CALL }\PY{n}{calc\PYZus{}E\PYZus{}NN}\PY{p}{(}\PY{p}{)}
  \PY{k}{ELSE}
    \PY{k}{CALL }\PY{n}{calc\PYZus{}Ewald\PYZus{}qe}\PY{p}{(}\PY{p}{)}
  \PY{k}{ENDIF}
\end{Verbatim}
\item
The following call will initialize various global variables (arrays) needed to
define Hamiltonian. It is mainly used for storing potential terms.

\begin{Verbatim}[commandchars=\\\{\}]
  \PY{k}{CALL }\PY{n}{alloc\PYZus{}hamiltonian}\PY{p}{(}\PY{p}{)}
\end{Verbatim}
\item
Allocate local pseudopotential. For periodic sinc LF the potential is constructed
directly on real space grid. For periodic LF, the potential is first constructed on
reciprocal space and then transformed to real space grid via inverse FFT.

\begin{Verbatim}[commandchars=\\\{\}]
  \PY{k}{IF}\PY{p}{(} \PY{n}{assume\PYZus{}isolated} \PY{o}{==} \PY{l+s+s1}{\PYZsq{}sinc\PYZsq{}} \PY{p}{)} \PY{k}{THEN}
    \PY{k}{CALL }\PY{n}{init\PYZus{}V\PYZus{}ps\PYZus{}loc}\PY{p}{(}\PY{p}{)}
  \PY{k}{ELSE}
    \PY{k}{CALL }\PY{n}{init\PYZus{}V\PYZus{}ps\PYZus{}loc\PYZus{}G}\PY{p}{(}\PY{p}{)}
  \PY{k}{ENDIF}
\end{Verbatim}
\item
Laplacian matrix $\nabla^2$ is initialized here.

\begin{Verbatim}[commandchars=\\\{\}]
  \PY{k}{CALL }\PY{n}{init\PYZus{}nabla2\PYZus{}sparse}\PY{p}{(}\PY{p}{)}
\end{Verbatim}
\item
Here we construct ILU0 preconditioner based on kinetic matrix.

\begin{Verbatim}[commandchars=\\\{\}]
  \PY{k}{CALL }\PY{n}{init\PYZus{}ilu0\PYZus{}prec}\PY{p}{(}\PY{p}{)}
\end{Verbatim}
\item
If the option \texttt{FREE\_NABLA2} is \texttt{.TRUE.}, then memory for storing
Laplacian matrix is freed immediately. In this way, application of kinetic operator
will be done by a matrix-free algorithm instead of using sparse-matrix multiplication.

\begin{Verbatim}[commandchars=\\\{\}]
  \PY{k}{IF}\PY{p}{(} \PY{n}{FREE\PYZus{}NABLA2} \PY{p}{)} \PY{k}{THEN}
    \PY{k}{CALL }\PY{n}{dealloc\PYZus{}nabla2\PYZus{}sparse}\PY{p}{(}\PY{p}{)}
  \PY{k}{ENDIF}
\end{Verbatim}
\item
This is a special setup for Poisson solver in the case of non-periodic
system. Currrently there are two different methods: (1) interpolating scaling
function (ISF) method and (2) Direct Algorithm for Gravitation and Electrostatic (DAGE).

\begin{Verbatim}[commandchars=\\\{\}]
  \PY{k}{IF}\PY{p}{(} \PY{n}{I\PYZus{}POISSON\PYZus{}SOLVE} \PY{o}{==} \PY{l+m+mi}{1} \PY{p}{)} \PY{k}{THEN}
    \PY{k}{CALL }\PY{n}{init\PYZus{}Poisson\PYZus{}solve\PYZus{}ISF}\PY{p}{(}\PY{p}{)}
  \PY{k}{ELSEIF}\PY{p}{(} \PY{n}{I\PYZus{}POISSON\PYZus{}SOLVE} \PY{o}{==} \PY{l+m+mi}{2} \PY{p}{)} \PY{k}{THEN}
    \PY{k}{CALL }\PY{n}{init\PYZus{}Poisson\PYZus{}solve\PYZus{}DAGE}\PY{p}{(}\PY{p}{)}
  \PY{k}{ENDIF}
\PY{k}{END }\PY{k}{SUBROUTINE}
\end{Verbatim}
\end{itemize}

\section{Subroutine \texttt{guess\_KS\_solutions}}

Generate guess solutions (density and orbitals) for \texttt{KS\_solve\_XXX}
subroutines.

Note that the global arrays \texttt{KS\_evals} and \texttt{KS\_evecs} are
allocated in this subroutine.

The algorithm used in this subroutine is not sophisticated. I have used ABINIT
routine \texttt{atmlength} to generate gaussian charge density.

For direct minimization (using \texttt{KS\_solve\_Emin\_XXX}),
we need to generate initial wavefunction, so an
iterative diagonalization step must be performed. This might cause significant
additional time before the actual direct minimization step starts.

For SCF, we can simply generate random initial wavefunction which will be used for
iterative diagonalization step.


\begin{Verbatim}[commandchars=\\\{\}]
\PY{k}{SUBROUTINE }\PY{n}{guess\PYZus{}KS\PYZus{}solutions}\PY{p}{(}\PY{p}{)}
\end{Verbatim}
Here are the imported variables:

\begin{Verbatim}[commandchars=\\\{\}]
  \PY{k}{USE }\PY{n}{m\PYZus{}input\PYZus{}vars}\PY{p}{,} \PY{k}{ONLY} \PY{p}{:} \PY{n}{startingwfc}
\end{Verbatim}
\begin{itemize}
\item
Generate gaussian charge density.


\begin{Verbatim}[commandchars=\\\{\}]
  \PY{c}{! FIXME: Need gen\PYZus{}guess\PYZus{}rho\PYZus{}gaussian for isolated system ?}
  \PY{c}{!        thus bypassing calculation of structure factors ?}
  \PY{k}{IF}\PY{p}{(} \PY{n}{startingwfc} \PY{o}{/}\PY{o}{=} \PY{l+s+s1}{\PYZsq{}random\PYZsq{}} \PY{p}{)} \PY{k}{THEN}
    \PY{k}{CALL }\PY{n}{gen\PYZus{}guess\PYZus{}rho\PYZus{}gaussian}\PY{p}{(}\PY{p}{)}
  \PY{k}{ENDIF}
\end{Verbatim}
\item
Generate random initial wavefunction.
This step is needed for both SCF and direct minimization method.

\begin{Verbatim}[commandchars=\\\{\}]
  \PY{k}{CALL }\PY{n}{gen\PYZus{}random\PYZus{}evecs}\PY{p}{(}\PY{p}{)}  \PY{c}{! also needed for initial diagonalization routine}
\end{Verbatim}
\item
If the input variable \texttt{startingwfc} is not set to \texttt{'random'}
then we need to do one iterative diagonalization step.
If it is set to random then we don't need to do iterative diagonalization
step as random wavefunction is suffice and subroutine will immediately return.

\begin{Verbatim}[commandchars=\\\{\}]
  \PY{k}{IF}\PY{p}{(} \PY{n}{startingwfc} \PY{o}{/}\PY{o}{=} \PY{l+s+s1}{\PYZsq{}random\PYZsq{}} \PY{p}{)} \PY{k}{THEN}
    \PY{c}{! This will call diagonalization routine}
    \PY{k}{CALL }\PY{n}{gen\PYZus{}gaussian\PYZus{}evecs}\PY{p}{(}\PY{p}{)}
  \PY{k}{ELSE}
    \PY{k}{WRITE}\PY{p}{(}\PY{o}{*}\PY{p}{,}\PY{o}{*}\PY{p}{)}
    \PY{k}{WRITE}\PY{p}{(}\PY{o}{*}\PY{p}{,}\PY{o}{*}\PY{p}{)} \PY{l+s+s1}{\PYZsq{}Using random starting wavefunction\PYZsq{}}
  \PY{k}{ENDIF}
\PY{k}{END }\PY{k}{SUBROUTINE}
\end{Verbatim}
\end{itemize}

\section{Subroutine \texttt{do\_KS\_solve}}

A driver routine for solving Kohn-Sham equations.


\begin{Verbatim}[commandchars=\\\{\}]
\PY{k}{SUBROUTINE }\PY{n}{do\PYZus{}KS\PYZus{}solve}\PY{p}{(}\PY{p}{)}
  \PY{k}{USE }\PY{n}{m\PYZus{}constants}\PY{p}{,} \PY{k}{ONLY} \PY{p}{:} \PY{n}{Ry2eV}
  \PY{k}{USE }\PY{n}{m\PYZus{}input\PYZus{}vars}\PY{p}{,} \PY{k}{ONLY} \PY{p}{:} \PY{n}{startingwfc}
  \PY{k}{USE }\PY{n}{m\PYZus{}options}\PY{p}{,} \PY{k}{ONLY} \PY{p}{:} \PY{n}{I\PYZus{}KS\PYZus{}Solve}
  \PY{k}{USE }\PY{n}{m\PYZus{}states}\PY{p}{,} \PY{k}{ONLY} \PY{p}{:} \PY{n}{Nstates}\PY{p}{,} \PY{n}{Focc}\PY{p}{,} \PY{p}{\PYZam{}}
                       \PY{n}{evals} \PY{o}{=}\PY{o}{\PYZgt{}} \PY{n}{KS\PYZus{}evals}\PY{p}{,} \PY{p}{\PYZam{}}
                       \PY{n}{evecs} \PY{o}{=}\PY{o}{\PYZgt{}} \PY{n}{KS\PYZus{}evecs}
  \PY{k}{IMPLICIT }\PY{k}{NONE} 
  \PY{k+kt}{INTEGER} \PY{k+kd}{::} \PY{n}{ist}
\end{Verbatim}

We are using direct minimization (this is the default):


\begin{Verbatim}[commandchars=\\\{\}]
  \PY{k}{IF}\PY{p}{(} \PY{n}{I\PYZus{}KS\PYZus{}SOLVE} \PY{o}{==} \PY{l+m+mi}{1} \PY{p}{)} \PY{k}{THEN} 
\end{Verbatim}
Call the main computational routine for direct minimization.
Note that for the moment the parameter $\alpha_t$ is hardcoded to $3\times10^{-5}$

\begin{Verbatim}[commandchars=\\\{\}]
    \PY{k}{CALL }\PY{n}{KS\PYZus{}solve\PYZus{}Emin\PYZus{}pcg}\PY{p}{(} \PY{l+m+mf}{3.}\PY{n}{d}\PY{o}{\PYZhy{}}\PY{l+m+mi}{5}\PY{p}{,} \PY{p}{.}\PY{n}{FALSE}\PY{p}{.} \PY{p}{)}
\end{Verbatim}
Display total energy components

\begin{Verbatim}[commandchars=\\\{\}]
    \PY{k}{CALL }\PY{n}{info\PYZus{}energies}\PY{p}{(}\PY{p}{)}
\end{Verbatim}
We need to calculate eigenvalues explicitly

\begin{Verbatim}[commandchars=\\\{\}]
    \PY{k}{CALL }\PY{n}{calc\PYZus{}evals}\PY{p}{(} \PY{n}{Nstates}\PY{p}{,} \PY{n}{Focc}\PY{p}{,} \PY{n}{evecs}\PY{p}{,} \PY{n}{evals} \PY{p}{)}
\end{Verbatim}

We are using the conventional SCF with electron density mixing


\begin{Verbatim}[commandchars=\\\{\}]
  \PY{k}{ELSEIF}\PY{p}{(} \PY{n}{I\PYZus{}KS\PYZus{}SOLVE} \PY{o}{==} \PY{l+m+mi}{2} \PY{p}{)} \PY{k}{THEN}
\end{Verbatim}
We need to calculate electron density and update the local potentials
if starting from random wavefunction

\begin{Verbatim}[commandchars=\\\{\}]
    \PY{k}{IF}\PY{p}{(} \PY{n}{startingwfc} \PY{o}{==} \PY{l+s+s1}{\PYZsq{}random\PYZsq{}} \PY{p}{)} \PY{k}{THEN}
      \PY{c}{! Initial Rhoe and potentials}
      \PY{k}{CALL }\PY{n}{calc\PYZus{}rhoe}\PY{p}{(} \PY{n}{Focc}\PY{p}{,} \PY{n}{evecs} \PY{p}{)}
      \PY{c}{! FIXME: need this ?}
      \PY{k}{CALL }\PY{n}{update\PYZus{}potentials}\PY{p}{(}\PY{p}{)}
    \PY{k}{ENDIF}
\end{Verbatim}
Call the main computational routine for SCF

\begin{Verbatim}[commandchars=\\\{\}]
    \PY{k}{CALL }\PY{n}{KS\PYZus{}solve\PYZus{}SCF}\PY{p}{(}\PY{p}{)}
    \PY{k}{CALL }\PY{n}{info\PYZus{}energies}\PY{p}{(}\PY{p}{)}
  \PY{k}{ENDIF}
\end{Verbatim}
Eigenvalues are displayed here

\begin{Verbatim}[commandchars=\\\{\}]
  \PY{k}{WRITE}\PY{p}{(}\PY{o}{*}\PY{p}{,}\PY{o}{*}\PY{p}{)}
  \PY{k}{WRITE}\PY{p}{(}\PY{o}{*}\PY{p}{,}\PY{o}{*}\PY{p}{)} \PY{l+s+s1}{\PYZsq{}Final eigenvalues (Ha and eV)\PYZsq{}}
  \PY{k}{WRITE}\PY{p}{(}\PY{o}{*}\PY{p}{,}\PY{o}{*}\PY{p}{)}
  \PY{k}{DO }\PY{n}{ist} \PY{o}{=} \PY{l+m+mi}{1}\PY{p}{,}\PY{n}{Nstates}
    \PY{k}{WRITE}\PY{p}{(}\PY{o}{*}\PY{p}{,}\PY{l+s+s1}{\PYZsq{}(1x,I8,2F18.10)\PYZsq{}}\PY{p}{)} \PY{n}{ist}\PY{p}{,} \PY{n}{evals}\PY{p}{(}\PY{n}{ist}\PY{p}{)}\PY{p}{,} \PY{n}{evals}\PY{p}{(}\PY{n}{ist}\PY{p}{)}\PY{o}{*}\PY{l+m+mf}{2.}\PY{n}{d0}\PY{o}{*}\PY{n}{Ry2eV}
  \PY{k}{ENDDO}
\end{Verbatim}
Write restart data

\begin{Verbatim}[commandchars=\\\{\}]
  \PY{c}{!FIXME Need tidy up}
  \PY{k}{CALL }\PY{n}{write\PYZus{}checkpoint}\PY{p}{(}\PY{p}{)}
  \PY{k}{CALL }\PY{n}{write\PYZus{}KS\PYZus{}evecs}\PY{p}{(}\PY{l+s+s1}{\PYZsq{}KS\PYZus{}evecs.dat\PYZsq{}}\PY{p}{)}
\PY{k}{END }\PY{k}{SUBROUTINE} 
\end{Verbatim}

\section{Subroutine \texttt{cleanup\_ffr\_LFDFT}}

This subroutine is the driver for various \texttt{dealloc\_XX} subroutines.

\begin{Verbatim}[commandchars=\\\{\}]
\PY{k}{SUBROUTINE }\PY{n}{cleanup\PYZus{}ffr\PYZus{}LFDFT}\PY{p}{(}\PY{p}{)}
\end{Verbatim}

Currently, only this option is needed.


\begin{Verbatim}[commandchars=\\\{\}]
  \PY{k}{USE }\PY{n}{m\PYZus{}options}\PY{p}{,} \PY{k}{ONLY} \PY{p}{:} \PY{n}{I\PYZus{}POISSON\PYZus{}SOLVE}
  \PY{k}{IMPLICIT }\PY{k}{NONE} 
\end{Verbatim}
This call will deallocate Kohn-Sham eigenvalues and eigenvectors

\begin{Verbatim}[commandchars=\\\{\}]
  \PY{k}{CALL }\PY{n}{dealloc\PYZus{}states}\PY{p}{(}\PY{p}{)}
  \PY{k}{IF}\PY{p}{(} \PY{n}{I\PYZus{}POISSON\PYZus{}SOLVE} \PY{o}{==} \PY{l+m+mi}{1} \PY{p}{)} \PY{k}{THEN}
    \PY{k}{CALL }\PY{n}{dealloc\PYZus{}Poisson\PYZus{}solve\PYZus{}ISF}\PY{p}{(}\PY{p}{)}
  \PY{k}{ELSEIF}\PY{p}{(} \PY{n}{I\PYZus{}POISSON\PYZus{}SOLVE} \PY{o}{==} \PY{l+m+mi}{2} \PY{p}{)} \PY{k}{THEN}
    \PY{k}{CALL }\PY{n}{dealloc\PYZus{}Poisson\PYZus{}solve\PYZus{}DAGE}\PY{p}{(}\PY{p}{)}
  \PY{k}{ENDIF}
  \PY{k}{CALL }\PY{n}{dealloc\PYZus{}nabla2\PYZus{}sparse}\PY{p}{(}\PY{p}{)}
  \PY{k}{CALL }\PY{n}{dealloc\PYZus{}ilu0\PYZus{}prec}\PY{p}{(}\PY{p}{)}
  \PY{k}{CALL }\PY{n}{dealloc\PYZus{}hamiltonian}\PY{p}{(}\PY{p}{)}
  \PY{k}{CALL }\PY{n}{dealloc\PYZus{}LF3d}\PY{p}{(}\PY{p}{)}
  \PY{k}{CALL }\PY{n}{dealloc\PYZus{}PsPot}\PY{p}{(}\PY{p}{)}
  \PY{k}{CALL }\PY{n}{dealloc\PYZus{}atoms}\PY{p}{(}\PY{p}{)}
\PY{k}{END }\PY{k}{SUBROUTINE} 
\end{Verbatim}

\section{Function \texttt{eval\_LF1d\_sinc}}

Evaluate Lagrange-sinc function:
\begin{equation}
\phi_{\alpha}(x) = \frac{1}{\sqrt{h}}
\frac{\sin\left[ \pi(x-x_{\alpha})/h \right]}{\pi(x-x_{\alpha})/h}
\end{equation}
where $h$ is grid spacing.


\begin{Verbatim}[commandchars=\\\{\}]
\PY{k}{FUNCTION }\PY{n}{eval\PYZus{}LF1d\PYZus{}sinc}\PY{p}{(} \PY{n}{N}\PY{p}{,} \PY{n}{grid}\PY{p}{,} \PY{n}{ibf}\PY{p}{,} \PY{n}{x} \PY{p}{)} \PY{k}{RESULT}\PY{p}{(}\PY{n}{ff}\PY{p}{)}
  \PY{k}{USE }\PY{n}{m\PYZus{}constants}\PY{p}{,} \PY{k}{ONLY} \PY{p}{:} \PY{n}{PI}
  \PY{k}{IMPLICIT }\PY{k}{NONE}
  \PY{c}{!}
  \PY{k+kt}{INTEGER} \PY{k+kd}{::} \PY{n}{N}
  \PY{k+kt}{REAL}\PY{p}{(}\PY{l+m+mi}{8}\PY{p}{)} \PY{k+kd}{::} \PY{n}{grid}\PY{p}{(}\PY{n}{N}\PY{p}{)}
  \PY{k+kt}{REAL}\PY{p}{(}\PY{l+m+mi}{8}\PY{p}{)} \PY{k+kd}{::} \PY{n}{ff}\PY{p}{,} \PY{n}{x}\PY{p}{,} \PY{n}{dx}
  \PY{k+kt}{INTEGER} \PY{k+kd}{::} \PY{n}{ibf}
  \PY{k+kt}{REAL}\PY{p}{(}\PY{l+m+mi}{8}\PY{p}{)}\PY{p}{,} \PY{k}{PARAMETER} \PY{k+kd}{::} \PY{n}{SMALL} \PY{o}{=} \PY{l+m+mf}{1.}\PY{n}{d}\PY{o}{\PYZhy{}}\PY{l+m+mi}{10}
  \PY{c}{!}
  \PY{k+kt}{REAL}\PY{p}{(}\PY{l+m+mi}{8}\PY{p}{)} \PY{k+kd}{::} \PY{n}{h}
  
  \PY{n}{h} \PY{o}{=} \PY{n}{grid}\PY{p}{(}\PY{l+m+mi}{2}\PY{p}{)} \PY{o}{\PYZhy{}} \PY{n}{grid}\PY{p}{(}\PY{l+m+mi}{1}\PY{p}{)}
  \PY{n}{dx} \PY{o}{=} \PY{n}{x} \PY{o}{\PYZhy{}} \PY{n}{grid}\PY{p}{(}\PY{n}{ibf}\PY{p}{)}
  \PY{k}{IF}\PY{p}{(} \PY{n+nb}{abs}\PY{p}{(}\PY{n}{dx}\PY{p}{)} \PY{o}{\PYZlt{}} \PY{n}{SMALL} \PY{p}{)} \PY{k}{THEN} 
    \PY{n}{dx} \PY{o}{=} \PY{n}{SMALL}
  \PY{k}{ENDIF} 
  \PY{n}{ff} \PY{o}{=} \PY{n+nb}{sin}\PY{p}{(} \PY{n}{PI}\PY{o}{*}\PY{n}{dx}\PY{o}{/}\PY{n}{h} \PY{p}{)} \PY{o}{/} \PY{p}{(}\PY{n}{PI}\PY{o}{*}\PY{n}{dx}\PY{p}{)} \PY{o}{*} \PY{n}{h} \PY{o}{/}\PY{n+nb}{sqrt}\PY{p}{(}\PY{n}{h}\PY{p}{)}
\PY{k}{END }\PY{k}{FUNCTION}
\end{Verbatim}

\section{Function \texttt{eval\_LF1d\_p}}

Evaluate periodic Lagrange function:
\begin{equation}
\phi_{\alpha}(x) = \frac{1}{\sqrt{NL}}
\sum_{i=1}^{N} \cos\left[k_{i}(x - x_{\alpha})\right]
\end{equation}
with $i = 1,2,\ldots,N$ and
\begin{equation}
k_{i} = \frac{2\pi(i - N' - 1)}{L}
\end{equation}
$N' = (N-1)/2$ and $N$ must be an odd number.


\begin{Verbatim}[commandchars=\\\{\}]
\PY{k}{FUNCTION }\PY{n}{eval\PYZus{}LF1d\PYZus{}p}\PY{p}{(}\PY{n}{N}\PY{p}{,} \PY{n}{L}\PY{p}{,} \PY{n}{grid}\PY{p}{,} \PY{n}{ibf}\PY{p}{,} \PY{n}{x}\PY{p}{)} \PY{k}{RESULT}\PY{p}{(}\PY{n}{ff}\PY{p}{)}
  \PY{k}{USE }\PY{n}{m\PYZus{}constants}\PY{p}{,} \PY{k}{ONLY} \PY{p}{:} \PY{n}{PI}
  \PY{k}{IMPLICIT }\PY{k}{NONE}
  \PY{c}{! arguments}
  \PY{k+kt}{INTEGER} \PY{k+kd}{::} \PY{n}{N}
  \PY{k+kt}{REAL}\PY{p}{(}\PY{l+m+mi}{8}\PY{p}{)} \PY{k+kd}{::} \PY{n}{L}
  \PY{k+kt}{REAL}\PY{p}{(}\PY{l+m+mi}{8}\PY{p}{)} \PY{k+kd}{::} \PY{n}{grid}\PY{p}{(}\PY{n}{N}\PY{p}{)}
  \PY{k+kt}{INTEGER} \PY{k+kd}{::} \PY{n}{ibf}
  \PY{k+kt}{REAL}\PY{p}{(}\PY{l+m+mi}{8}\PY{p}{)} \PY{k+kd}{::} \PY{n}{x}\PY{p}{,} \PY{n}{ff}
  \PY{c}{!}
  \PY{k+kt}{REAL}\PY{p}{(}\PY{l+m+mi}{8}\PY{p}{)} \PY{k+kd}{::} \PY{n}{pre1}
  \PY{k+kt}{INTEGER} \PY{k+kd}{::} \PY{n}{ii}
  \PY{c}{!}
  \PY{n}{pre1} \PY{o}{=} \PY{l+m+mf}{1.}\PY{n}{d0}\PY{o}{/}\PY{n+nb}{sqrt}\PY{p}{(}\PY{n}{N}\PY{o}{*}\PY{n}{L}\PY{p}{)}
  \PY{n}{ff} \PY{o}{=} \PY{l+m+mf}{0.}\PY{n}{d0}
  \PY{k}{DO }\PY{n}{ii} \PY{o}{=} \PY{l+m+mi}{1}\PY{p}{,}\PY{n}{N}
    \PY{n}{ff} \PY{o}{=} \PY{n}{ff} \PY{o}{+} \PY{n+nb}{cos}\PY{p}{(} \PY{n}{PI}\PY{o}{*}\PY{p}{(} \PY{l+m+mi}{2}\PY{o}{*}\PY{n}{ii} \PY{o}{\PYZhy{}} \PY{n}{N} \PY{o}{\PYZhy{}} \PY{l+m+mi}{1} \PY{p}{)}\PY{o}{*}\PY{p}{(} \PY{n}{x} \PY{o}{\PYZhy{}} \PY{n}{grid}\PY{p}{(}\PY{n}{ibf}\PY{p}{)} \PY{p}{)}\PY{o}{/} \PY{n}{L} \PY{p}{)}
  \PY{k}{ENDDO}
  \PY{n}{ff} \PY{o}{=} \PY{n}{ff}\PY{o}{*}\PY{n}{pre1}
\PY{k}{END }\PY{k}{FUNCTION}
\end{Verbatim}

\section{Function \texttt{eval\_LF1d\_c}}

Evaluate cluster Lagrange function:
\begin{equation}
\phi_{\alpha}(x) = \frac{2}{\sqrt{(N+1)L}}
\sum_{i=1}^{N} \sin\left(k_{i}x\right) \sin\left(k_{i}x_{\alpha}\right)
\end{equation}
with $i = 1,2,\ldots,N$ and
\begin{equation}
k_{i} = \frac{\pi i}{L}
\end{equation}


\begin{Verbatim}[commandchars=\\\{\}]
\PY{k}{FUNCTION }\PY{n}{eval\PYZus{}LF1d\PYZus{}c}\PY{p}{(} \PY{n}{N}\PY{p}{,} \PY{n}{L}\PY{p}{,} \PY{n}{A}\PY{p}{,} \PY{n}{grid}\PY{p}{,} \PY{n}{ibf}\PY{p}{,} \PY{n}{x} \PY{p}{)} \PY{k}{RESULT}\PY{p}{(}\PY{n}{ff}\PY{p}{)}
  \PY{k}{USE }\PY{n}{m\PYZus{}constants}\PY{p}{,} \PY{k}{ONLY} \PY{p}{:} \PY{n}{PI}
  \PY{k}{IMPLICIT }\PY{k}{NONE}
  \PY{c}{!}
  \PY{k+kt}{INTEGER} \PY{k+kd}{::} \PY{n}{N}
  \PY{k+kt}{REAL}\PY{p}{(}\PY{l+m+mi}{8}\PY{p}{)} \PY{k+kd}{::} \PY{n}{L}
  \PY{k+kt}{REAL}\PY{p}{(}\PY{l+m+mi}{8}\PY{p}{)} \PY{k+kd}{::} \PY{n}{A}
  \PY{k+kt}{REAL}\PY{p}{(}\PY{l+m+mi}{8}\PY{p}{)} \PY{k+kd}{::} \PY{n}{grid}\PY{p}{(}\PY{n}{N}\PY{p}{)}
  \PY{k+kt}{REAL}\PY{p}{(}\PY{l+m+mi}{8}\PY{p}{)} \PY{k+kd}{::} \PY{n}{ff}
  \PY{k+kt}{INTEGER} \PY{k+kd}{::} \PY{n}{ibf}
  \PY{k+kt}{REAL}\PY{p}{(}\PY{l+m+mi}{8}\PY{p}{)} \PY{k+kd}{::} \PY{n}{x}
  \PY{c}{!}
  \PY{k+kt}{REAL}\PY{p}{(}\PY{l+m+mi}{8}\PY{p}{)} \PY{k+kd}{::} \PY{n}{ki}\PY{p}{,} \PY{n}{pre}
  \PY{k+kt}{INTEGER} \PY{k+kd}{::} \PY{n}{ii}
  \PY{n}{pre} \PY{o}{=} \PY{l+m+mf}{2.}\PY{n}{d0}\PY{o}{/}\PY{n+nb}{sqrt}\PY{p}{(} \PY{p}{(}\PY{n}{N}\PY{o}{+}\PY{l+m+mi}{1}\PY{p}{)}\PY{o}{*}\PY{n}{L} \PY{p}{)}
  \PY{n}{ff} \PY{o}{=} \PY{l+m+mf}{0.}\PY{n}{d0}
  \PY{k}{DO }\PY{n}{ii} \PY{o}{=} \PY{l+m+mi}{1}\PY{p}{,} \PY{n}{N}
    \PY{n}{ki} \PY{o}{=} \PY{n}{PI}\PY{o}{*}\PY{n}{ii}\PY{o}{/}\PY{n}{L}
    \PY{n}{ff} \PY{o}{=} \PY{n}{ff} \PY{o}{+} \PY{n+nb}{sin}\PY{p}{(} \PY{n}{ki}\PY{o}{*}\PY{p}{(}\PY{n}{grid}\PY{p}{(}\PY{n}{ibf}\PY{p}{)} \PY{o}{\PYZhy{}} \PY{n}{A}\PY{p}{)} \PY{p}{)} \PY{o}{*} \PY{n+nb}{sin}\PY{p}{(} \PY{n}{ki}\PY{o}{*}\PY{p}{(}\PY{n}{x} \PY{o}{\PYZhy{}} \PY{n}{A}\PY{p}{)} \PY{p}{)}
  \PY{k}{ENDDO}
  \PY{n}{ff} \PY{o}{=} \PY{n}{ff}\PY{o}{*}\PY{n}{pre}
\PY{k}{END }\PY{k}{FUNCTION}
\end{Verbatim}

\end{document}

