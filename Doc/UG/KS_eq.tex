\section{Kohn-Sham equation}

Within LDA, Kohn-Sham energy functional can be written as:
\begin{equation}
E_{\mathrm{LDA}}\left[\{\psi_{i}(\mathbf{r})\}\right] = 
E_{\mathrm{kin}} + E_{\mathrm{ion}} + E_{\mathrm{Ha}} + E_{\mathrm{xc}}
\end{equation}
with the following energy terms.

(1) kinetic energy:
\begin{equation}
E_{\mathrm{kin}} = -\frac{1}{2}\sum_{i} \int
\psi_{i}^{*}(\mathbf{r})\,\nabla^2\,\psi_{i}(\mathbf{r})
\,\mathrm{d}\mathbf{r}
\end{equation}

(2) ion-electron interaction energy:
\begin{equation}
E_{\mathrm{ion}} = \int V_{\mathrm{ion}}(\mathbf{r})\, \rho(\mathbf{r})\,
\mathrm{d}\mathbf{r}
\end{equation}

(3) Hartree (electrostatic) energy:
\begin{equation}
E_{\mathrm{Ha}} = \int \frac{1}{2}
\dfrac{\rho(\mathbf{r})\rho(\mathbf{r}')}
{\left|\mathbf{r} - \mathbf{r}'\right|}
\mathrm{d}\mathbf{r}\mathrm{d}\mathbf{r}'
\end{equation}

(4) Exchange-correlation energy (using LDA):
\begin{equation}
E_{\mathrm{xc}} = \int \epsilon_{\mathrm{xc}}\left[\rho(\mathbf{r})\right]
\rho(\mathbf{r})\,\mathrm{d}\mathbf{r}
\end{equation}

Central to the density functional theory is the so-called Kohn-Sham
equation.
This equation can be written as:
\begin{equation}
\left[
-\frac{1}{2}\nabla^2  + V_{\mathrm{KS}}(\mathbf{r})
\right] \psi_{i}(\mathbf{r}) =
\epsilon_{i}\psi_{i}(\mathbf{r})
\end{equation}
where $\epsilon{i}$ and $\psi_{i}(\mathbf{r})$ is known as Kohn-Sham
eigenvalues and eigenvectors (orbitals).
Quantity $V_{\mathrm{KS}}$ is called the Kohn-Sham potential, which can be
written as sum of several potentials:
\begin{equation}
V_{\mathrm{KS}}(\mathbf{r}) = V_{\mathrm{ion}}(\mathbf{r}) + V_{\mathrm{Ha}}(\mathbf{r})
+ V_{\mathrm{xc}}(\mathbf{r})
\label{eq:KS-pot}
\end{equation}

$V_{\mathrm{ion}}$ denotes attractive potential between ion (or atomic nuclei)
with electrons. This potential can be written as:
\begin{equation}
V_{\mathrm{ion}}(\mathrm{r}) =
\sum_{I}^{N_{\mathrm{atoms}}}
\frac{Z_{I}}{ \left| \mathbf{r} - \mathbf{R}_{I} \right| }
\end{equation}
This potential is Coulombic and has singularities
at the ionic centers. It is generally difficult to describe this
potential fully. It is common to replace the full Coulombic potential
with softer potential which is known as pseudopotential.
There are various types or flavors of pseudopotentials.
In the current implementation, ion-electron potential, $V_{\mathrm{ion}}$
is treated by pseudopotential. HGH-type pseudopotential is employed due to the
the availability of analytic forms both in real and reciprocal space.

$V_{\mathrm{Ha}}$ is the classical Hartree potential. It is defined as
\begin{equation}
V_{\mathrm{Ha}}(\mathbf{r}) = \int
\frac{\rho(\mathbf{r}')}
{\mathbf{r} - \mathbf{r}'}\,\mathrm{d}\mathbf{r}',
\end{equation}
where $\rho(\mathbf{r})$ denotes electronic density:
\begin{equation}
\rho(\mathbf{r}) = \sum_{i}^{N_{\mathrm{occ}}}
\psi^{*}_{i}(\mathbf{r}) \psi_{i}(\mathbf{r})
\end{equation}
Alternatively, Hartree potential can also be obtained via solving Poisson equation:
\begin{equation}
\nabla^{2} V_{\mathrm{Ha}}(\mathbf{r}) = -4\pi \rho(\mathbf{r})
\end{equation}

The last term in Equation \eqref{eq:KS-pot} is exchange-correlation potential.

