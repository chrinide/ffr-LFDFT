\subsection{Description of LF basis set}

Description of LF basis set in 3d is given in module
{\tt m\_LF3d}. All global variables in this module is given
prefix {\tt LF3d}.

\begin{fortrancode}
MODULE m_LF3d
  IMPLICIT NONE
  INTEGER, PARAMETER :: LF3d_PERIODIC = 1
  INTEGER, PARAMETER :: LF3d_CLUSTER  = 2
  INTEGER, PARAMETER :: LF3d_SINC     = 3
  INTEGER :: LF3d_TYPE
  INTEGER, DIMENSION(3) :: LF3d_NN
  REAL(8), DIMENSION(3) :: LF3d_LL
  REAL(8), DIMENSION(3) :: LF3d_AA, LF3d_BB
  REAL(8), DIMENSION(3) :: LF3d_hh
  INTEGER :: LF3d_Npoints
  REAL(8) :: LF3d_dVol
  REAL(8), ALLOCATABLE :: LF3d_grid_x(:)
  REAL(8), ALLOCATABLE :: LF3d_grid_y(:)
  REAL(8), ALLOCATABLE :: LF3d_grid_z(:)
  REAL(8), ALLOCATABLE :: LF3d_D1jl_x(:,:)
  REAL(8), ALLOCATABLE :: LF3d_D1jl_y(:,:)
  REAL(8), ALLOCATABLE :: LF3d_D1jl_z(:,:)
  REAL(8), ALLOCATABLE :: LF3d_D2jl_x(:,:)
  REAL(8), ALLOCATABLE :: LF3d_D2jl_y(:,:)
  REAL(8), ALLOCATABLE :: LF3d_D2jl_z(:,:)
  REAL(8), ALLOCATABLE :: LF3d_lingrid(:,:)
  INTEGER, ALLOCATABLE :: LF3d_xyz2lin(:,:,:)
  INTEGER, ALLOCATABLE :: LF3d_lin2xyz(:,:)
  REAL(8), ALLOCATABLE :: LF3d_G2(:)
  REAL(8), ALLOCATABLE :: LF3d_Gv(:,:)
END MODULE
\end{fortrancode}

Variables in {\tt m\_LF3d} is initialized by calling the subroutine
{\tt init\_LF3d\_XX()}, where {\tt XX} may be one of:
\begin{itemize}
\item {\tt p}: periodic LFF
\item {\tt c}: cluster LF
\item {\tt sinc}: sinc L
\end{itemize}

\begin{fortrancode}
SUBROUTINE init_LF3d_p( NN, AA, BB )
SUBROUTINE init_LF3d_c( NN, AA, BB )
SUBROUTINE init_LF3d_sinc( NN, hh )
\end{fortrancode}

In the above subroutines:
\begin{itemize}
\item {\tt NN}: an array of 3 integers, specifying sampling points in $x$,
$y$ and $z$ direction.
\item {\tt AA}: an array of 3 floats, specifying left ends of unit cell.
\item {\tt BB}: an array of 3 floats, specifying right ends of unit cell.
\item {\tt hh}: an array of 3 floats, specifying spacing between adjacent sampling points.
\end{itemize}

Note that for periodic and cluster LF we have to specify {\tt NN}, {\tt AA}, and {\tt BB}
while for sinc LF we have to specify {\tt NN} and {\tt hh}.
Note that for periodic LF {\tt NN} must be odd numbers.

Example:
\begin{fortrancode}
NN = (/ 35, 35, 35 /)
AA = (/ 0.d0, 0.d0, 0.d0 /)
BB = (/ 6.d0, 6.d0, 6.d0 /)
CALL init_LF3d_p( NN, AA, BB )
\end{fortrancode}
