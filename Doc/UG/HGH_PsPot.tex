\section{HGH pseudopotential}

HGH pseudopotential has analytic forms both in real space
and reciprocal space.

Local component of pseudopotential in real space
\begin{multline}
V_{\mathrm{loc}}(\mathbf{r}) = 
-\dfrac{Z_{\mathrm{ion}}}{r}
\mathrm{erf}\left(
\dfrac{r}{\sqrt{2}r_{\mathrm{loc}}}
\right) + \\
\exp
\left[ -\frac{1}{2}
\left( \frac{r}{r_{\mathrm{loc}}}\right)^2
\right]
\times
\left[
C_{1} +
C_{2}\left( \frac{r}{r_{\mathrm{loc}}}\right)^2 +
C_{3}\left( \frac{r}{r_{\mathrm{loc}}}\right)^4 +
C_{4}\left( \frac{r}{r_{\mathrm{loc}}}\right)^6
\right]
\end{multline}
with parameters: $r_{\mathrm{loc}}$, $C_{1}$, $C_{2}$, $C_{3}$, and $C_{4}$.

Local component of local pseudopotential in $\mathbf{G}$-space:
\begin{multline}
V_{\mathrm{loc}}(\mathbf{G}) = 
-\dfrac{1}{\Omega}
\dfrac{4\pi Z_{\mathrm{ion}}}{G^2}
\exp\left[
-\dfrac{1}{2}
\left(Gr_{\mathrm{loc}}\right)^2
\right] + 
\sqrt{8\pi^3}\dfrac{r_{\mathrm{loc}}}{\Omega}
\exp\left[
-\dfrac{1}{2}
\left(Gr_{\mathrm{loc}}\right)^2
\right] \times \\
\left\{
C_{1}
+ C_{2}\left[3-\left(Gr_{\mathrm{loc}}\right)^2\right]
+ C_{3}\left[15 - 10\left(Gr_{\mathrm{loc}}\right)^2
  \left(Gr_{\mathrm{loc}}\right)^4 \right] \right. \\
\left.
+ C_{4}\left[105 - 105\left(Gr_{\mathrm{loc}}\right)^2
  + 21\left(Gr_{\mathrm{loc}}\right)^4
  - \left(Gr_{\mathrm{loc}}\right)^6\right]
\right\}
\end{multline}

Nonlocal component of pseudopotential can be written as
\begin{equation}
V_{l}(\mathbf{r},\mathbf{r}') =
\sum_{i=1}^{3} \sum_{j=1}^{3} \sum_{m=-l}^{l}
\beta_{ilm}(\mathbf{r})\,h^{l}_{ij}\,\beta^{*}_{jlm}(\mathbf{r}')
\end{equation}
with atomic-centered functions projector functions as
\begin{equation}
\beta_{ilm}(\mathbf{r}) = 
p^{l}_{i}(r) Y_{lm}(\hat{\mathbf{r}})
\end{equation}
The radial projector functions have the following form in real space
\begin{equation}
p_{i}^{l}(r) = \frac{\sqrt{2} r^{l+2(i-1)}
\exp\left( -\dfrac{r^2}{2r_{l}^2} \right) }
{r_{l}^{l+(4i-1)/2}
\sqrt{\Gamma\left(l + \dfrac{4i-1}{2}\right)}
}
\end{equation}

The radial projector functions satisfy the following normalization condition
\begin{equation}
\int_{0}^{\infty} p_{i}^{l}(r) p_{i}^{l}(r)\, r^2\,\mathrm{d}r = 1
\end{equation}

For $l = 0$, the Fourier transform of radial projector functions can be written as:
\begin{align}
p^{l=0}_{1}(G) & = \dfrac
{4\sqrt{2r_0^3}\pi^{5/4}}
{\sqrt{\Omega}\exp\left[(Gr_0)^2/2\right]}
\\
p^{l=0}_{2}(G) & = \dfrac
{\sqrt{8\dfrac{2r_0^3}{15}}\pi^{5/4}
\left( 3-(Gr_0)^2 \right)}
{\sqrt{\Omega}\exp\left[(Gr_0)^2/2\right]}
\\
p^{l=0}_{3}(G) & = \dfrac
{16\sqrt{\dfrac{2r_0^3}{105}}\pi^{5/4}
\left(15 - 10(Gr_0)^2 - (Gr_0)^4 \right)}
{3\sqrt{\Omega}\exp\left[(Gr_0)^2/2\right]}
\end{align}

For $l=1$, the Fourier transform of radial projector functions can be written as
\begin{align}
p^{l=1}_{1}(G) & = \dfrac
{8\sqrt{\dfrac{r_1^5}{3}}\pi^{5/4}G}
{\sqrt{\Omega}\exp\left[(Gr_1)^2/2\right]}
\\
p^{l=1}_{2}(G) & = \dfrac
{16\sqrt{\dfrac{r_1^5}{105}}\pi^{5/4}
\left( 5 - (Gr_1)^2 \right)G}
{\sqrt{\Omega}\exp\left[(Gr_1)^2/2\right]}
\\
p^{l=1}_{3}(G) & = \dfrac
{32\sqrt{\dfrac{r_1^5}{1155}}\pi^{5/4}
\left( 35 - 14(Gr_1)^2 + (Gr_1)^4 \right)G}
{3\sqrt{\Omega}\exp\left[(Gr_1)^2/2\right]}
\end{align}


For $l=2$, the Fourier transform of radial projector functions can be written as
\begin{align}
p^{l=2}_{1}(G) & = \dfrac
{8\sqrt{\dfrac{2r_2^7}{15}}\pi^{5/4}G^2}
{\sqrt{\Omega}\exp\left[(Gr_2)^2/2\right]}
\\
p^{l=2}_{2}(G) & = \dfrac
{16\sqrt{\dfrac{2r_2^7}{105}}\pi^{5/4}
\left( 7 - (Gr_2)^2 \right)G^2}
{3\sqrt{\Omega}\exp\left[(Gr_2)^2/2\right]}
\end{align}

For $l=3$, the Fourier transform of radial projector function can be written as
\begin{equation}
p^{l=3}_{1}(G) = \dfrac
{16\sqrt{\dfrac{2r_3^9}{105}}\pi^{5/4}G^3}
{\sqrt{\Omega}\exp\left[(Gr_3)^2/2\right]}
\end{equation}


