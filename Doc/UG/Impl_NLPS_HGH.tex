\subsection{Nonlocal pseudopotential}

Nonlocal HGH pseudopotential action can be defined as follows:
\begin{equation}
\hat{V}_{\mathrm{NL}} \psi =
\end{equation}

Action of nonlocal pseudopotential to wavefunction:

\begin{fortrancode}
SUBROUTINE op_V_ps_NL( Nstates, Vpsi )
  USE m_LF3d, ONLY : Npoints => LF3d_Npoints
  USE m_PsPot, ONLY : NbetaNL, betaNL, prj2beta, Ps => Ps_HGH_Params
  USE m_atoms, ONLY : Natoms, atm2species
  USE m_hamiltonian, ONLY : betaNL_psi
  IMPLICIT NONE
  INTEGER :: Nstates
  REAL(8) :: Vpsi(Npoints,Nstates)
  INTEGER :: ia, isp, ist, ibeta, jbeta, iprj, jprj
  INTEGER :: l, m
  REAL(8) :: hij

  IF( NbetaNL <= 0 ) THEN
    RETURN
  ENDIF

  Vpsi(:,:) = 0.d0

  DO ist = 1,Nstates
    DO ia = 1,Natoms
      isp = atm2species(ia)
      DO l = 0,Ps(isp)%lmax
      DO m = -l,l
        DO iprj = 1,Ps(isp)%Nproj_l(l)
        DO jprj = 1,Ps(isp)%Nproj_l(l)
          ibeta = prj2beta(iprj,ia,l,m)
          jbeta = prj2beta(jprj,ia,l,m)
          hij = Ps(isp)%h(l,iprj,jprj)
          Vpsi(:,ist) = Vpsi(:,ist) + hij*betaNL(:,ibeta)*betaNL_psi(ia,ist,jbeta)
        ENDDO ! jprj
        ENDDO ! iprj
      ENDDO ! m
      ENDDO ! l
    ENDDO
  ENDDO

END SUBROUTINE
\end{fortrancode}

The array {\tt betaNL} is defined initialized in subroutine {\tt init\_betaNL}:

\begin{fortrancode}
SUBROUTINE init_betaNL()

  USE m_LF3d, ONLY : Npoints => LF3d_Npoints, &
                     lingrid => LF3d_lingrid, &
                     LL => LF3d_LL, &
                     dVol => LF3d_dVol
  USE m_PsPot, ONLY : betaNL, NbetaNL, &
                      Ps => Ps_HGH_Params
  USE m_atoms, ONLY : atpos => AtomicCoords, Natoms, atm2species
  USE m_Ps_HGH, ONLY : hgh_eval_proj_R
  IMPLICIT NONE
  INTEGER :: ia, isp, l, m, iprj
  INTEGER :: Np_beta, ip, ibeta
  REAL(8) :: dr_vec(3)
  REAL(8) :: dr
  REAL(8) :: Ylm_real
  REAL(8) :: nrm

  ALLOCATE( betaNL(Npoints,NbetaNL) )

  ! loop structure must be the same as in init_PsPot
  ibeta = 0
  DO ia = 1,Natoms
    isp = atm2species(ia)
    DO l = 0,Ps(isp)%lmax
      DO iprj = 1,Ps(isp)%Nproj_l(l)
        DO m = -l,l
          ibeta = ibeta + 1
          Np_beta = 0
          DO ip = 1,Npoints
            CALL calc_dr_periodic_1pnt( LL, atpos(:,ia), lingrid(:,ip), dr_vec )
            dr = sqrt( dr_vec(1)**2 + dr_vec(2)**2 + dr_vec(3)**2 )
            IF( dr <= Ps(isp)%rcut_NL(l) ) THEN
              Np_beta = Np_beta + 1
              betaNL(ip,ibeta) = hgh_eval_proj_R( Ps(isp), l, iprj, dr ) * Ylm_real( l, m, dr_vec )
            ENDIF
          ENDDO
          nrm = sum(betaNL(:,ibeta)**2)*dVol
          WRITE(*,'(1x,A,I5,I8,F18.10)') 'ibeta, Np_beta, integ = ', ibeta, Np_beta, nrm
        ENDDO ! iprj
      ENDDO ! m
    ENDDO ! l
  ENDDO

END SUBROUTINE
\end{fortrancode}


and {\tt betaNL\_psi} is calculated in {\tt calc\_betaNL\_psi}:

\begin{fortrancode}
SUBROUTINE calc_betaNL_psi( Nstates, psi )

  USE m_LF3d, ONLY : Npoints => LF3d_Npoints, &
                       dVol => LF3d_dVol
  USE m_PsPot, ONLY : NbetaNL, betaNL
  USE m_hamiltonian, ONLY : betaNL_psi
  USE m_atoms, ONLY : Natoms
  IMPLICIT NONE
  INTEGER :: Nstates
  REAL(8) :: psi(Npoints,Nstates)
  INTEGER :: ist, ibeta, ia
  REAL(8) :: ddot

  ! immediate return if no projectors are available
  IF( NbetaNL <= 0 ) THEN
    RETURN
  ENDIF

  betaNL_psi(:,:,:) = 0.d0

  DO ia = 1,Natoms
    DO ist = 1,Nstates
      DO ibeta = 1,NbetaNL
        betaNL_psi(ia,ist,ibeta) = ddot( Npoints, betaNL(:,ibeta),1, psi(:,ist),1 ) * dVol
      ENDDO
    ENDDO
  ENDDO

END SUBROUTINE
\end{fortrancode}
