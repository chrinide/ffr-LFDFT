\subsection{Main program}

Currently, the calculation flow of the main program of \ffrLFDFT is as follows:
\begin{itemize}
\item Getting program argument as input file and reading the input file
\item Initializing molecular structure, pseudopotentials, and 
Lagrange basis functions, including grids
\item Setting additional options if necessary based on the input file
\item Initializing electronic states variables
\item Setting up Hamiltonian: potential and kinetic operators.
\item Solving the Kohn-Sham equation via direct minimization
or self-consistent field
\end{itemize}

The appropriate subroutine calls is given below.

\begin{fortrancode}
CALL getarg( 1, filein )
CALL read_input( filein )
CALL setup_from_input()
CALL setup_options()

CALL init_betaNL()
CALL init_states()
CALL init_strfact_shifted()
CALL calc_Ewald_qe()
CALL alloc_hamiltonian()
CALL init_V_ps_loc_G()
CALL init_nabla2_sparse()
CALL init_ilu0_prec()
CALL gen_guess_rho_gaussian()
ALLOCATE( evecs(Npoints,Nstates), evals(Nstates) )
CALL gen_random_evecs()
CALL gen_gaussian_evecs()

IF( I_KS_SOLVE == 1 ) THEN 
  CALL KS_solve_Emin_pcg()
  CALL calc_evals( Nstates, Focc, evecs, evals )
ELSEIF( I_KS_SOLVE == 2 ) THEN 
  CALL KS_solve_SCF_v2()
ENDIF 
\end{fortrancode}


